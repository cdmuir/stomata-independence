% Options for packages loaded elsewhere
\PassOptionsToPackage{unicode}{hyperref}
\PassOptionsToPackage{hyphens}{url}
%
\documentclass[
  12pt,
]{article}
\usepackage{amsmath,amssymb}
\usepackage{lmodern}
\usepackage{setspace}
\usepackage{iftex}
\ifPDFTeX
  \usepackage[T1]{fontenc}
  \usepackage[utf8]{inputenc}
  \usepackage{textcomp} % provide euro and other symbols
\else % if luatex or xetex
  \usepackage{unicode-math}
  \defaultfontfeatures{Scale=MatchLowercase}
  \defaultfontfeatures[\rmfamily]{Ligatures=TeX,Scale=1}
\fi
% Use upquote if available, for straight quotes in verbatim environments
\IfFileExists{upquote.sty}{\usepackage{upquote}}{}
\IfFileExists{microtype.sty}{% use microtype if available
  \usepackage[]{microtype}
  \UseMicrotypeSet[protrusion]{basicmath} % disable protrusion for tt fonts
}{}
\makeatletter
\@ifundefined{KOMAClassName}{% if non-KOMA class
  \IfFileExists{parskip.sty}{%
    \usepackage{parskip}
  }{% else
    \setlength{\parindent}{0pt}
    \setlength{\parskip}{6pt plus 2pt minus 1pt}}
}{% if KOMA class
  \KOMAoptions{parskip=half}}
\makeatother
\usepackage{xcolor}
\usepackage[margin=1in]{geometry}
\usepackage{longtable,booktabs,array}
\usepackage{calc} % for calculating minipage widths
% Correct order of tables after \paragraph or \subparagraph
\usepackage{etoolbox}
\makeatletter
\patchcmd\longtable{\par}{\if@noskipsec\mbox{}\fi\par}{}{}
\makeatother
% Allow footnotes in longtable head/foot
\IfFileExists{footnotehyper.sty}{\usepackage{footnotehyper}}{\usepackage{footnote}}
\makesavenoteenv{longtable}
\usepackage{graphicx}
\makeatletter
\def\maxwidth{\ifdim\Gin@nat@width>\linewidth\linewidth\else\Gin@nat@width\fi}
\def\maxheight{\ifdim\Gin@nat@height>\textheight\textheight\else\Gin@nat@height\fi}
\makeatother
% Scale images if necessary, so that they will not overflow the page
% margins by default, and it is still possible to overwrite the defaults
% using explicit options in \includegraphics[width, height, ...]{}
\setkeys{Gin}{width=\maxwidth,height=\maxheight,keepaspectratio}
% Set default figure placement to htbp
\makeatletter
\def\fps@figure{htbp}
\makeatother
\setlength{\emergencystretch}{3em} % prevent overfull lines
\providecommand{\tightlist}{%
  \setlength{\itemsep}{0pt}\setlength{\parskip}{0pt}}
\setcounter{secnumdepth}{-\maxdimen} % remove section numbering
\newlength{\cslhangindent}
\setlength{\cslhangindent}{1.5em}
\newlength{\csllabelwidth}
\setlength{\csllabelwidth}{3em}
\newlength{\cslentryspacingunit} % times entry-spacing
\setlength{\cslentryspacingunit}{\parskip}
\newenvironment{CSLReferences}[2] % #1 hanging-ident, #2 entry spacing
 {% don't indent paragraphs
  \setlength{\parindent}{0pt}
  % turn on hanging indent if param 1 is 1
  \ifodd #1
  \let\oldpar\par
  \def\par{\hangindent=\cslhangindent\oldpar}
  \fi
  % set entry spacing
  \setlength{\parskip}{#2\cslentryspacingunit}
 }%
 {}
\usepackage{calc}
\newcommand{\CSLBlock}[1]{#1\hfill\break}
\newcommand{\CSLLeftMargin}[1]{\parbox[t]{\csllabelwidth}{#1}}
\newcommand{\CSLRightInline}[1]{\parbox[t]{\linewidth - \csllabelwidth}{#1}\break}
\newcommand{\CSLIndent}[1]{\hspace{\cslhangindent}#1}
% \documentclass[11pt]{article} % redundant
\usepackage[sc]{mathpazo} %Like Palatino with extensive math support
\usepackage{fullpage}
\usepackage[authoryear,sectionbib,sort]{natbib}
\linespread{1.7}
\usepackage[utf8]{inputenc}
\usepackage{lineno}
\usepackage{titlesec}
\usepackage{multirow} % for merging rows in tables
\usepackage{caption} % for \captionsetup

% flush left while keep identation
\makeatletter
\newcommand\iraggedright{%
  \let\\\@centercr\@rightskip\@flushglue \rightskip\@rightskip
  \leftskip\z@skip}
\makeatother

\titleformat{\section}[block]{\Large\bfseries\filcenter}{\thesection}{1em}{}
\titleformat{\subsection}[block]{\Large\itshape\filcenter}{\thesubsection}{1em}{}
\titleformat{\subsubsection}[block]{\large\itshape}{\thesubsubsection}{1em}{}
\titleformat{\paragraph}[runin]{\itshape}{\theparagraph}{1em}{}[. ]\renewcommand{\refname}{Literature Cited}

\title{How important are functional and developmental constraints on phenotypic evolution? An empirical test with the stomatal anatomy of flowering plants}

\setlength{\parindent}{5ex}

\author{}
\date{\vspace{-2.5em}}
\ifLuaTeX
  \usepackage{selnolig}  % disable illegal ligatures
\fi
\IfFileExists{bookmark.sty}{\usepackage{bookmark}}{\usepackage{hyperref}}
\IfFileExists{xurl.sty}{\usepackage{xurl}}{} % add URL line breaks if available
\urlstyle{same} % disable monospaced font for URLs
\hypersetup{
  hidelinks,
  pdfcreator={LaTeX via pandoc}}

\author{}
\date{}

\begin{document}



\setstretch{1}
\begin{flushleft}
{\Large
\textbf\newline{How important are functional and developmental constraints on phenotypic evolution? An empirical test with the stomatal anatomy of flowering plants}
}
\newline
\\
Christopher D. Muir\textsuperscript{1,*} (ORCID: 0000-0003-2555-3878),
Miquel \`{A}ngel Conesa\textsuperscript{2},
Jeroni Galm\'{e}s\textsuperscript{2} (ORCID: 0000-0002-7299-9349),
Varsha S. Pathare\textsuperscript{3} (ORCID: 0000-0001-6220-7531),
Patricia Rivera\textsuperscript{4},
Rosana López Rodríguez\textsuperscript{5},
Teresa Terrazas\textsuperscript{4},
Dongliang Xiong\textsuperscript{6} (ORCID: 0000-0002-6332-2627)
\\
\bigskip
\bf{1} School of Life Sciences, University of Hawaii at M\=anoa, Honolulu, HI 96822, USA \\
\bf{2} Research Group on Plant Biology under Mediterranean Conditions, Departament de Biologia, Universitat de les Illes Balears, Ctra. Valldemossa km 7.5, E-07122, Palma, Spain \\
\bf{3} School of Biological Sciences, Washington State University, Pullman, WA 99164-4236, USA \\
\bf{4} Departamento de Botánica, Instituto de Biología, Universidad Nacional Autónoma de México, Apartado Postal 70‑367, 04510 Mexico City, Mexico \\
\bf{5} Departamento de Sistemas y Recursos Naturales, Universidad Politécnica de Madrid, 28040 Madrid, Spain \\
\bf{6} National Key Laboratory of Crop Genetic Improvement, MOA Key Laboratory of Crop Ecophysiology and Farming System in the Middle Reaches of the Yangtze River, College of Plant Science and Technology, Huazhong Agricultural University, Wuhan, Hubei 430070, China\\
\bigskip
$\ast$ cdmuir@hawaii.edu

\textit{Keywords}: adaptation, amphistomy, developmental integration, leaf, packing limits, phylogenetic comparative methods, stomata

\end{flushleft}

\newpage{}

\hypertarget{abstract}{%
\section{Abstract}\label{abstract}}

Quantifying the relative contribution of functional and developmental constraints on phenotypic variation is a longstanding goal of macroevolution, but it is often difficult to distinguish different types of constraints. Alternatively, selection can limit phenotypic (co)variation if some trait combinations are generally maladaptive. The anatomy of leaves with stomata on both surfaces (amphistomatous) present a unique opportunity to test the importance of functional and developmental constraints on phenotypyic evolution. The key insight is that stomata on each leaf surface encounter the same functional and developmental constraints, but potentially different selective pressures because of leaf asymmetry in light capture, gas exchange, and other features. Independent evolution of stomatal traits on each surface imply that functional and developmental constraints alone likely do not explain trait covariance. Packing limits on how many stomata can fit into a finite epidermis and cell-size-mediated developmental integration are hypothesized to constrain variation in stomatal anatomy. The simple geometry of the planar leaf surface and knowledge of stomatal development make it possible to derive equations for phenotypic (co)variance caused by these constraints and compare them with data. We analyzed evolutionary covariance between stomatal density and length in amphistomatous leaves from 236 phylogenetically independent contrasts using a robust Bayesian model. Stomatal anatomy on each surface diverges partially independently, meaning that packing limits and developmental integration are not sufficient to explain phenotypic (co)variation. Hence, (co)variation in ecologically important traits like stomata arises in part because there is a limited range of evolutionary optima. We show how it is possible to evaluate the contribution of different constraints by deriving expected patterns of (co)variance and testing them using similar but separate tissues, organs, or sexes.\\

\hypertarget{introduction}{%
\section{Introduction}\label{introduction}}

Selection should move populations toward their multivariate phenotypic optimum if traits can evolve independently and there is sufficient genetic variation. Yet divergence in one trait often covaries with other traits and covariance can persist for millions of years (\protect\hyperlink{ref-schluter_adaptive_1996}{Schluter 1996}). Covariance is ``a rough local measure of the strength of constraint'' (\protect\hyperlink{ref-maynard_smith_developmental_1985}{Maynard Smith et al. 1985}) that can be broken down into functional and developmental constraints. Functional constraints are ``limitations imposed by time, energy, or the laws of physics'' (\protect\hyperlink{ref-arnold_constraints_1992}{Arnold 1992}). In other words, certain trait combinations are not physically or geometrically possible. A classic example is shell coiling among invertebrate lineages in which the morphospace of possible phenotypes is constrained by hard geometrical limits (\protect\hyperlink{ref-raup_geometric_1966}{Raup 1966}; \protect\hyperlink{ref-mcghee_theoretical_1999}{McGhee 1999}). Within the space of possible phenotypes, developmental constraints can ``bias\ldots the production of variant phenotypes or {[}place{]} a limitation on phenotypic variability'' (\protect\hyperlink{ref-maynard_smith_developmental_1985}{Maynard Smith et al. 1985}). For example, Fibonacci leaf arrangement (phyllotaxis) may arise from a packing constraint of primordia on the developing apex (\protect\hyperlink{ref-mitchison_phyllotaxis_1977}{Mitchison 1977}; \protect\hyperlink{ref-reinhardt_law_2022}{Reinhardt and Gola 2022}; but see \protect\hyperlink{ref-niklas_role_1988}{Niklas 1988} for an adaptive explanation). The adaptationist view is that all adaptive areas of trait space are occupied and that natural selection constrains phenotypic variation because maladaptive forms cannot evolve. Under this view, interspecific trait covariation evolves because `missing', maladaptive trait combinations are nonrandomly distributed in trait space. For example, a negative correlation between apparency and toxicity can arise because prey can avoid predation by being cryptic or toxic, but no prey population can exist for long if it is both apparent and palatable. Understanding phenotypic constraint is challenging, but a useful starting point is determining whether phenotypic covariation can be explained by functional or developmental constraints (\protect\hyperlink{ref-mcghee_theoretical_1999}{McGhee 1999}, \protect\hyperlink{ref-mcghee_geometry_2007}{2007}; \protect\hyperlink{ref-olson_plant_2019}{Olson 2019}). If phenotypic covariation is inconsistent with functional and developmental constraints, this provides a strong impetus to test adaptive explanations. This is sometimes referred to as selective constraint because selection eliminates phenotypic variation that is generally maladaptive.

In this study, we will address packing problems and developmental integration, specific forms of functional and developmental constraint relevant to our study system, stomatal anatomy. We introduce these concepts generally in this paragraph. Packing a number objects into a finite space is a common functional constraint on organisms. Regular geometries that appear in nature such as helices and hexagons (think DNA and honeycombs) are often optimal solutions to packing problems (\protect\hyperlink{ref-mackenzie_proving_1999}{Mackenzie 1999}; \protect\hyperlink{ref-maritan_optimal_2000}{Maritan et al. 2000}). Notice that functional constraint does not preclude selection, but the presence of a packing limit changes the range of possible phenotypes. Developmental integration is a form of developmental constraint on multivariate phenotypic evolution and we use these terms interchangeably in this study. Developmentally integrated traits have a ``disposition for covariation'' (\protect\hyperlink{ref-armbruster_integrated_2014}{Armbruster et al. 2014}), meaning that evolutionary divergence between lineages in one trait will be tightly associated with divergence in another trait. Allometry is a classic, albeit contested, example of developmental integration that may constrain phenotypic evolution (reviewed in \protect\hyperlink{ref-pelabon_evolution_2014}{Pélabon et al. 2014}). Strong allometric covariation between traits within populations can constrain macroevolutionary divergence for long periods of time depending on the strength and direction of selection (\protect\hyperlink{ref-lande_quantitative_1979}{Lande 1979}). However, developmental integration does not necessarily hamper adaptation, and can even accelerate adaptive evolution when trait covariation is aligned with the direction of selection (\protect\hyperlink{ref-hansen_is_2003}{Hansen 2003}). For example, fusion of floral parts increases their developmental integration which may increase the rate and precision of multivariate adaptation to specialist pollinators (Berg's rule, \protect\hyperlink{ref-berg_general_1959}{Berg 1959}, \protect\hyperlink{ref-berg_ecological_1960}{1960}; \protect\hyperlink{ref-conner_raissa_2014}{Conner and Lande 2014}; \protect\hyperlink{ref-armbruster_covariance_1999}{Armbruster et al. 1999}).

Stomatal anatomy on the leaves of flowering plants provides an exceptional opportunity to test for functional and developmental constraints because 1) there are 1000s of species to compare and 2) the main packing constraints and developmental steps are analytically tractable. This means it is possible to derive quantitative predictions and test their generality using large comparative data sets representing millions of years of evolutionary history. For this purpose, a heretofore unappreciated fact about stomata is that many leaves have stomata on both lower and upper surfaces. The packing and developmental constraints are the same for stomata on each surface, but the selection may differ. Therefore, if packing and developmental constraints dominate, stomatal anatomy on each surface should diverge in concert. Failure to do so implies that independent evolution is possible and that selection against generally maladaptive phenotypes explains at least some of the covariance between traits. Phylogenetic comparisons of stomatal anatomy provide a statistically powerful, general, and elegant way to distinguish different phenotypic constraints that would be impossible in many other traits with as much ecological significance. The next sections provide background information on stomatal anatomy, how it varies, and why functional or developmental constraints might be important.

\hypertarget{how-stomatal-anatomy-varies-and-why-it-matters}{%
\subsection{How stomatal anatomy varies and why it matters}\label{how-stomatal-anatomy-varies-and-why-it-matters}}

Stomata are microscopic pores formed by a pair of guard cells that regulate gas exchange (CO\(_2\) gain and water vapor loss) on the leaves or other photosynthetic surfaces of most land plants. Stomata originated once in the history of land plants around 500 Ma, diversified rapidly in density and size, and have been maintained in most lineages except some bryophytes and aquatic plants (recently reviewed in \protect\hyperlink{ref-clark_origin_2022}{Clark et al. 2022}). Stomata respond physiologically by opening and closing in response to light, humidity, temperature, circadian rhythm, and plant water status (\protect\hyperlink{ref-hetherington_role_2003}{Hetherington and Woodward 2003}; \protect\hyperlink{ref-lawson_guard_2020}{Lawson and Matthews 2020}). The stomatal size, density, and distribution on a mature leaf do not change, so the maximum rate of gas exchange is fixed. However, the plant may respond plastically to environmental cues such as light and CO\(_2\) by altering stomatal anatomy in new leaves (\protect\hyperlink{ref-casson_influence_2008}{Casson and Gray 2008}). Physiological responses (aperture change) and plastic responses (new leaves with changed anatomy) may be alternative strategies for plants to acclimate to environmental change (\protect\hyperlink{ref-haworth_co-ordination_2013}{Haworth, Elliott-Kingston, and McElwain 2013}). Finally, stomatal anatomy can evolve due to inherited changes in stomatal development. Plastic and genetic changes in stomatal anatomy are both ecologically important, but most studies do not use a common garden design that would tease apart their relative contribution.

We focus on anatomical variation in the density, size, and patterning of stomata on a leaf because these factors set the maximum stomatal conductance to CO\(_2\) diffusing into a leaf and the amount of water that transpires from it (\protect\hyperlink{ref-sack_hydrology_2003}{Sack et al. 2003}; \protect\hyperlink{ref-franks_effect_2001}{Franks and Farquhar 2001}; \protect\hyperlink{ref-galmes_leaf_2013}{Galmés et al. 2013}; \protect\hyperlink{ref-harrison_influence_2020}{Harrison et al. 2020}). Plants typically operate below their anatomical maximum by dynamically regulating stomatal aperture. Even though operational stomatal conductance determines the realized photosynthetic rate and water-use efficiency, anatomical parameters are useful in that they set the range of stomatal function (\protect\hyperlink{ref-de_boer_optimal_2016}{de Boer et al. 2016}) and are correlated with actual stomatal function under natural conditions (\protect\hyperlink{ref-murray_consistent_2020}{Murray et al. 2020}). All else being equal, larger, more densely packed, but evenly spaced stomata increase gas exchange (\protect\hyperlink{ref-franks_maximum_2009}{Franks and Beerling 2009}; \protect\hyperlink{ref-dow_physiological_2014}{Dow, Berry, and Bergmann 2014}; \protect\hyperlink{ref-lehmann_effects_2015}{Lehmann and Or 2015}). Smaller stomata may also be able to respond more rapidly than larger stomata, proving the ability of leaves to track short duration environmental change (\protect\hyperlink{ref-drake_smaller_2013}{Drake, Froend, and Franks 2013}). Stomata are most often found only on the lower leaf surface (hypostomy), but occur on both surfaces (amphistomy) in some species (\protect\hyperlink{ref-metcalfe_anatomy_1950}{Metcalfe and Chalk 1950}; \protect\hyperlink{ref-parkhurst_adaptive_1978}{Parkhurst 1978}; \protect\hyperlink{ref-mott_adaptive_1982}{Mott, Gibson, and O'Leary 1982}). Amphistomatous leaves have a second parallel pathway from the substomatal cavities through the leaf internal airspace to sites of carboxylation in the mesophyll (\protect\hyperlink{ref-parkhurst_adaptive_1978}{Parkhurst 1978}; \protect\hyperlink{ref-gutschick_photosynthesis_1984}{Gutschick 1984}). Thus amphistomatous leaves have lower resistance to diffusion through the airspace which increases the photosynthetic rate (\protect\hyperlink{ref-parkhurst_intercellular_1990}{Parkhurst and Mott 1990}). If total stomatal and other conductances to CO\(_2\) supply could be held constant, then an amphistomatous leaf will have a greater conductance than an otherwise identical hypostomatous leaf. The magnitude of the advantage depends on the resistance to diffusion through the internal airspace, which is variable among species.

The adaptive significance and ecological distribution of leaves with different stomatal anatomies is complex and there is much yet to learn. Seed plants posses a wider range of stomatal anatomies than ferns and fern allies, which are restricted to having large stomata, at low density, only on the lower surface (\protect\hyperlink{ref-de_boer_optimal_2016}{de Boer et al. 2016}). In general, trees and shrubs have greater stomatal density than herbs, but there is a of lot variation within growth forms depending on the ecological niche (\protect\hyperlink{ref-salisbury_i_1928}{Salisbury 1928}; \protect\hyperlink{ref-kelly_plant_1995}{Kelly and Beerling 1995}). A commonly observed trend is that leaves from higher light environments tend to have greater stomatal density (\protect\hyperlink{ref-salisbury_i_1928}{Salisbury 1928}; \protect\hyperlink{ref-mott_adaptive_1982}{Mott, Gibson, and O'Leary 1982}; \protect\hyperlink{ref-gibson_structure-function_1996}{Gibson 1996}; \protect\hyperlink{ref-smith_associations_1998}{W. K. Smith, Bell, and Shepherd 1998}; \protect\hyperlink{ref-jordan_using_2014}{Jordan, Carpenter, and Brodribb 2014}; \protect\hyperlink{ref-muir_making_2015}{Muir 2015}; \protect\hyperlink{ref-bucher_stomatal_2017}{Bucher et al. 2017}). This may explain why, perhaps counterintuitively, plants in dry environments tend to have more stomata. Drier habitats are more open, enabling plants with higher stomatal density to photosynthesize more when water is available, but close stomata during drought (\protect\hyperlink{ref-liu_variation_2018}{Liu et al. 2018}). Over recent human history, stomatal density has tended to decline within species as atmospheric CO\(_2\) concentrations have risen (\protect\hyperlink{ref-woodward_stomatal_1987}{Woodward 1987}; \protect\hyperlink{ref-royer_stomatal_2001}{Royer 2001}). It is unclear whether most of this change is plastic or genetic, but the overall direction is consistent with the hypothesis that plants decrease gas exchange as CO\(_2\) availability increases.

Stomatal ratio is a continuous measure of how stomata are deployed across leaf surfaces. The adaptive significance of variation in stomatal ratio is uncertain, but we have some clues based on the distribution of hypo- and amphistomatous leaves. Despite the fact that amphistomy can increase photosynthesis, most leaves are hypostomatous. Amphistomy should increase photosynthesis most under saturating-light conditions where CO\(_2\) supply limits photosynthesis. However, the light environment alone cannot explain why hypostomatous leaves predominate in shade plants (\protect\hyperlink{ref-muir_is_2019}{Muir 2019}), suggesting that we need to understand the costs of upper stomata better. One factor may be increased vulnerability to pathogens. For example, upper stomata increase susceptibility to rust pathogens in \emph{Populus} (\protect\hyperlink{ref-mckown_association_2014}{McKown et al. 2014}, \protect\hyperlink{ref-mckown_role_2019}{2019}; \protect\hyperlink{ref-fetter_growthdefense_2021}{Fetter, Nelson, and Keller 2021}). Amphistomy may also cause the palisade mesophyll to dry out under strong vapor pressure deficits (\protect\hyperlink{ref-buckley_how_2015}{Buckley et al. 2015}). Other hypotheses about the adaptive significance of stomatal ratio are discussed in Muir (\protect\hyperlink{ref-muir_making_2015}{2015}) and Drake et al. (\protect\hyperlink{ref-drake_two_2019}{2019}).

\hypertarget{major-features-of-stomatal-anatomical-macroevolution}{%
\subsection{Major features of stomatal anatomical macroevolution}\label{major-features-of-stomatal-anatomical-macroevolution}}

Two major features of stomatal anatomy have been recognized for decades but we do not yet understand the evolutionary forces that generate and maintain them. We refer to these features as ``inverse size-density scaling'' and ``bimodal stomatal ratio'' (Fig. \ref{fig:concepts}). Inverse size-density scaling refers to the negative interspecific correlation between the size of the stomatal apparatus and the density of stomata (\protect\hyperlink{ref-weiss_untersuchungen_1865}{Weiss 1865}; \protect\hyperlink{ref-franks_maximum_2009}{Franks and Beerling 2009}; \protect\hyperlink{ref-de_boer_optimal_2016}{de Boer et al. 2016}; \protect\hyperlink{ref-sack_developmental_2016}{Sack and Buckley 2016}; \protect\hyperlink{ref-liu_scaling_2021}{Liu et al. 2021}). Across species, leaves with smaller stomata tend to pack them more densely, but there is significant variation about this general trend (Fig. \ref{fig:concepts}a). Stomatal size and density determine the maximum stomatal conductance to CO\(_2\) and water vapor but also take up space on the epidermis, which could be costly for both construction and maintenance. Natural selection should favor leaves that have enough stomata of sufficient size to supply CO\(_2\) for photosynthesis. Hence leaves with few, small stomata and high photosynthetic rates do not exist because they would not supply enough CO\(_2\). Conversely, excess stomata or extra large stomata beyond the optimum may result in stomatal interference where the CO\(_2\) concentration gradient around one stomate merges with that of its neighbor (\protect\hyperlink{ref-zeiger_stomatal_1987}{Zeiger, Farquhar, and Cowan 1987}; \protect\hyperlink{ref-lehmann_effects_2015}{Lehmann and Or 2015}), incur metabolic costs (\protect\hyperlink{ref-deans_optimization_2020}{Deans et al. 2020}), and/or risk hydraulic failure (\protect\hyperlink{ref-henry_stomatal_2019}{Henry et al. 2019}). The distribution of stomatal size and density may therefore represent the combinations that ensure enough, but not too much, stomatal conductance. Franks and Beerling (\protect\hyperlink{ref-franks_maximum_2009}{2009}) further hypothesized that the evolution of small stomata in angiosperms enabled increased stomatal conductance while minimizing the epidermal area allocated to stomata.

A striking feature of the interspecific variation in stomatal ratio is that trait values are not uniformly distributed, but strongly bimodal (Fig. \ref{fig:concepts}b). Bimodal stomatal ratio refers to the observation that the ratio of stomatal density on the adaxial (upper) surface to the density on the abaxial (lower) has distinct modes (Fig. \ref{fig:concepts}b). Amphistomy occurs most often in herbaceous plants from open, high light habitats (\protect\hyperlink{ref-salisbury_i_1928}{Salisbury 1928}; \protect\hyperlink{ref-mott_adaptive_1982}{Mott, Gibson, and O'Leary 1982}; \protect\hyperlink{ref-gibson_structure-function_1996}{Gibson 1996}; \protect\hyperlink{ref-smith_associations_1998}{W. K. Smith, Bell, and Shepherd 1998}; \protect\hyperlink{ref-jordan_using_2014}{Jordan, Carpenter, and Brodribb 2014}; \protect\hyperlink{ref-muir_making_2015}{Muir 2015}, \protect\hyperlink{ref-muir_light_2018}{2018}; \protect\hyperlink{ref-bucher_stomatal_2017}{Bucher et al. 2017}). Muir (\protect\hyperlink{ref-muir_making_2015}{2015}) described bimodal stomatal ratio formally but the pattern is apparent in earlier comparative studies of the British flora (\emph{cf.} \protect\hyperlink{ref-peat_comparative_1994}{Peat and Fitter 1994, fig. 1}).

\hypertarget{packing-limits-developmental-integration-and-stomatal-anatomy}{%
\subsection{Packing limits, developmental integration, and stomatal anatomy}\label{packing-limits-developmental-integration-and-stomatal-anatomy}}

Given the significance of stomata for plant function and global vegetation modeling (\protect\hyperlink{ref-berry_stomata:_2010}{Berry, Beerling, and Franks 2010}), we would like to understand what factors constrain their anatomical variation. Here we focus on interspecific variation in mean trait values rather than intraspecific variation. Packing constraints and developmental integration could explain inverse size-density scaling. The density and size of stomata on a planar leaf surface can be viewed as a packing problem where the total area allocated to stomata cannot exceed the total leaf area. This is a functional, or geometric, constraint because certain combinations of large size and high density are not physically possible. This functional constraint cannot explain why combinations of low density and small size are rare, but may explain why stomatal size must decrease when density increases as the leaf runs out of space. The packing limit of functional stomata is less than the entire leaf area, but the exact value is unclear. The realized upper limit is close to 1/3 or 1/2 (\protect\hyperlink{ref-de_boer_optimal_2016}{de Boer et al. 2016}; \protect\hyperlink{ref-sack_developmental_2016}{Sack and Buckley 2016}; \protect\hyperlink{ref-liu_scaling_2021}{Liu et al. 2021}) for the species' mean, not an individual leaf.

Guard cell size and spacing between stomata (the inverse of density) are developmentally intertwined because guard cells and epidermal pavement cells between stomata develop from the same meristem. Before guard cell meristemoids form via asymmetric cell division (\protect\hyperlink{ref-dow_patterning_2014}{Dow and Bergmann 2014}), the size of guard and epidermal cells are influenced by meristematic cell volume and expansion. Evolutionary shifts in meristematic cell volume or expansion rate could cause both increased stomatal size and lower density because epidermal cells between stomata are larger (\protect\hyperlink{ref-brodribb_unified_2013}{Brodribb, Jordan, and Carpenter 2013}). For example, larger genomes increase meristematic cell volume (\protect\hyperlink{ref-simova_geometrical_2012}{Šímová and Herben 2012}), which sets a lower bound on final cell volume. Although different expansion rates in guard and epidermal pavement cells can reduce the correlation in their final size, the fact that species with larger genomes tend toward having larger stomata and lower density may indicate an effect of development integration on stomatal anatomy (\protect\hyperlink{ref-beaulieu_genome_2008}{Beaulieu et al. 2008}; \protect\hyperlink{ref-simonin_genome_2018}{Simonin and Roddy 2018}; \protect\hyperlink{ref-roddy_scaling_2020}{Roddy et al. 2020}). Developmental integration in this case would not necessarily slow adaptive evolution if the main axes of selection were aligned with the developmental correlation. For example, if higher maximum stomatal conductance were achieved primarily by increasing stomatal density and decreasing stomatal size as proposed by Franks and Beerling (\protect\hyperlink{ref-franks_maximum_2009}{2009}), then developmental integration might accelerate the response to selection compared to a case where stomatal size and density are completely independent.

Muir (\protect\hyperlink{ref-muir_making_2015}{2015}) showed that bimodality can arise adaptively if fitness optima are restricted to separate regimes. Specifically, if either the fitness costs or benefits of allocating additional stomata to the upper surface are nonlinear, with the correct curvature, then there are distinct ``hypostomatous'' and ``amphistomatous'' regimes. An alternative hypothesis is that stomatal traits on the ab- and adaxial surfaces are developmentally integrated because stomatal development is regulated the same way on each surface. In hypostomatous leaves, stomatal development is turned off in the adaxial surface. In amphistomatous leaves, stomatal development proceeds on both surfaces, but evolutionary changes in stomatal development affect traits on both surfaces because they are tethered by a shared developmental program. This is a developmental constraint because the fact that stomatal development is the same on each surface constrains the type of variation available for selection. Developmental integration would lead to a bimodal trait distribution because leaves would either be hypostomatous (stomatal ratio equal to 0) or have similar densities on each surface (stomatal ratio approximately 0.5). To our knowledge, this hypothesis has not been put forward in the literature.

\hypertarget{hypotheses-and-predictions}{%
\subsection{Hypotheses and predictions}\label{hypotheses-and-predictions}}

The overarching question is whether major features of stomatal anatomy in terrestrial angiosperms are consistent with packing constraints and/or developmental integration mediated by cell size. Since stomata on both surfaces of amphistomatous leaves are subject to the same functional and developmental constraints, if these constraints are most important we predict similar patterns of trait covariation on abaxial and adaxial surfaces. Conversely, if traits covary differently on each surface it would indicate that stomatal anatomical traits can evolve independently and selection against generally maladaptive trait combinations shapes covariation. Analogously, variation in the genetic correlation and interspecific divergence of sexually dimorphic traits in dioecious species demonstrate that integration is not fixed and can be modified by selection (\protect\hyperlink{ref-barrett_sexual_2013}{Barrett and Hough 2013}). We framed specific hypotheses and predictions around how functional or developmental constraints might explain either inverse size-density scaling or bimodal stomatal ratio.

\noindent {\bf{Inverse size-density scaling}}

Both packing limits and developmental integration could contribute to inverse size-density scaling. If limits on the fraction of epidermal area occupied by stomata constrains the combinations of stomatal size and density that are evolutionarily accessible, then we predict that evolutionary divergence in stomatal size and/or density will decrease as the fraction of epidermal area occupied by stomata increases. Furthermore, if divergence slows as epidermal area occupied by stomata because of a packing limit, it should slow down the same way for both ab- and adaxial surfaces.

The second hypothesis is that cell size mediates developmental integration between stomatal size and density. If developmental integration is the primary reason for inverse size-density scaling, then amphistomatous leaves will exhibit identical size-density scaling on each surface. If the stomatal size and density scale differently on each surface, this implies that they can evolve independently and that selection against generally maladaptive trait combinations shapes their covariance. Furthermore, we predicted that divergence in genome size, which is strongly associated with meristematic cell volume (\protect\hyperlink{ref-simova_geometrical_2012}{Šímová and Herben 2012}), would covary with stomatal size and density similarly on each surface.

\noindent {\bf{Bimodal stomatal ratio}}

If the developmental integration hypothesis is correct, it also implies stomatal size and density will diverge in concert on each surface because the developmental function is fixed. Therefore we predict that divergence of stomatal traits on one surface will be isometric with divergence in stomatal traits on the other surface. This type of developmental integration limits the expression of variation and could give rise to a bimodal stomatal ratio. Suppose that in hypostomatous leaves, stomatal development is completely suppressed. In amphistomatous leaves, stomatal development proceeds identically on each surface because the developmental function is identical. This would lead to a tendency for equal density on each surface.

We formalized these hypotheses into a mathematical framework to derive quantitative predictions that we tested in a phylogenetic comparative framework by compiling stomatal anatomy data from the literature for a broad range of flowering plants.

\hypertarget{materials-and-methods}{%
\section{Materials and Methods}\label{materials-and-methods}}

Unless otherwise mentioned, we performed all data wrangling and statistical analyses in \emph{R} version 4.2.2 (\protect\hyperlink{ref-r_core_team_r:_2022}{R Core Team 2022}). Source code is publicly available on GitHub (\url{https://github.com/cdmuir/stomata-independence}) and archived on Zenodo (see Data availability).

\hypertarget{theory-divergence-with-and-without-developmental-constraint}{%
\subsection{Theory: divergence with and without developmental constraint}\label{theory-divergence-with-and-without-developmental-constraint}}

Developmental integration could shape patterns of phenotypic macroevolution, but a major hindrance to progress is that verbal models do not make precise, quantitative predictions that distinguish it from alternatives. An advantage of testing developmental integration in stomata is that their development is well studied (\protect\hyperlink{ref-bergmann_stomatal_2007}{Bergmann and Sack 2007}; \protect\hyperlink{ref-dow_patterning_2014}{Dow and Bergmann 2014}; \protect\hyperlink{ref-sack_developmental_2016}{Sack and Buckley 2016}). We can leverage that knowledge to build a developmental function and derive equations for phenotypic (co)variance caused by developmental integration. If observed patterns of evolution are inconsistent with developmental integration, theory may also help identify which parameters lead to developmental disintegration. We combined and extended stomatal development models to predict how stomatal density and length would diverge if stomatal development were constrained and how those predictions would change if stomatal development were unconstrained. We summarize our methods verbally here and direct readers interested in the mathematical details to the online supplement. A graphical summary is provided in Fig. \ref{fig:developmental-integration}. We imposed constraint by assuming the stomatal developmental function is constrained. The developmental function maps cell size prior to differentiation onto stomatal size and density using two parameters. The first parameter describes how cell volume is apportioned between epidermal cells and guard cell meristemoids during asymmetric cell division (\protect\hyperlink{ref-dow_patterning_2014}{Dow and Bergmann 2014}). The second parameter is stomatal index (\protect\hyperlink{ref-salisbury_i_1928}{Salisbury 1928}; \protect\hyperlink{ref-sack_developmental_2016}{Sack and Buckley 2016}), which is determined by amplifying and spacing divisions after asymmetric cell division (\protect\hyperlink{ref-dow_patterning_2014}{Dow and Bergmann 2014}). When these parameters are fixed, divergence in stomatal size and density is determined by divergence in meristematic cell volume and expansion prior to asymmetric division. We relaxed this constraint by treating parameters of the developmental function as random variables that can diverge between species. We used random variable algebra to derive predicted (co)variance in divergence between stomatal length and density (see \protect\hyperlink{ref-lynch_genetics_1998}{Lynch and Walsh 1998} for key random variable algebra theorems).

\hypertarget{data-synthesis}{%
\subsection{Data synthesis}\label{data-synthesis}}

We searched the literature for studies that measured stomatal density and stomatal size, either guard cell length or stomatal pore length, for both abaxial and adaxial leaf surfaces. In other words, we did not include studies unless they reported separate density and size values for each surface. We did not record leaf angle because it is typically not reported, but we presume that for the vast majority of taxa that the abaxial is the lower surface and the adaxial is the upper surface. This is reversed in resupinate leaves, but to the best of our knowledge, our synthesis did not include resupinate leaves. None of the species with resupinate leaves listed by Chitwood et al. (\protect\hyperlink{ref-chitwood_conflict_2012}{2012}) are in our data set. We refer to guard cell length as stomatal length and converted stomatal pore length to stomatal length assuming guard cell length is twice pore length (\protect\hyperlink{ref-sack_developmental_2016}{Sack and Buckley 2016}). Table \ref{tab:traits} lists anatomical traits and symbols. Abaxial traits are subscripted with \(_{\textrm{ab}}\); adaxial traits are subscripted with \(_{\textrm{ad}}\)

Data on stomatal anatomy are spread over a disparate literature and we have not attempted an exhaustive synthesis of amphistomatous leaf stomatal anatomy. We began our search by reviewing papers that cited key studies of amphistomy (\protect\hyperlink{ref-parkhurst_adaptive_1978}{Parkhurst 1978}; \protect\hyperlink{ref-mott_adaptive_1982}{Mott, Gibson, and O'Leary 1982}; \protect\hyperlink{ref-muir_making_2015}{Muir 2015}). We supplemented these by searching Clarivate \emph{Web of Science} for ``guard cell length'' because most studies that report guard cell length also report stomatal density, whereas the reverse is not true. We identified additional studies by reviewing the literature cited of papers we found and through opportunistic discovery. The final data set contained 5104 observations of stomatal density and length from 1242 taxa and 38 primary studies (Table \ref{tab:sources}). However, many of these data were excluded if taxonomic name and phylogenetic placement could not be resolved (see below). Finally, we included some unpublished data. Stomatal size data were collected on grass species described in Pathare, Koteyeva, and Cousins (\protect\hyperlink{ref-pathare_increased_2020}{2020}). We also included unpublished data on 14 amphistomatous wild tomato species (\emph{Solanum} sect. \emph{Lycopersicum} and sect. \emph{Lycopersicoides}) grown in pots under outdoor summer Mediterranean conditions as described in Muir, Galmés, and Conesa (\protect\hyperlink{ref-muir_unpublished_2022}{2022}). We took ab- and adaxial epidermal imprints using clear nail polish of the mid-portion of the lamina away from major veins on the terminal leaflet of the youngest, fully expanded leaf from 1-5 replicates per taxon. With a brightfield light microscope, we counted stomata in three 0.571 mm\(^2\) fields of view and divided by the total area to estimate density. We measured the average guard cell length of 60 stomata, 20 per field of view, to estimate stomatal size. The data set is publicly available as an \emph{R} package \textbf{ropenstomata} (\url{https://github.com/cdmuir/ropenstomata}). We collected data on genome size from the Angiosperm DNA C-values database (\protect\hyperlink{ref-leitch_angiosperm_2019}{Leitch et al. 2019}; \protect\hyperlink{ref-pellicer_plant_2020}{Pellicer and Leitch 2020}). When multiple ploidy levels were available for a taxon, we chose the lowest one for consistency.

\hypertarget{phylogenetically-independent-contrasts}{%
\subsection{Phylogenetically independent contrasts}\label{phylogenetically-independent-contrasts}}

We generated an ultrametric, bifurcating phylogeny of 638 taxa by resolving and removing ambiguous taxonomic names, placing taxa on the GBOTB.extended mega-tree of seed plants (\protect\hyperlink{ref-smith_constructing_2018}{S. A. Smith and Brown 2018}; \protect\hyperlink{ref-zanne_three_2014}{Zanne et al. 2014}), and resolving polytomies using published sequence data. Divergence times in this phylogeny are based on extensive fossil calibration (see \protect\hyperlink{ref-smith_constructing_2018}{S. A. Smith and Brown 2018}; \protect\hyperlink{ref-magallon_metacalibrated_2015}{Magallón et al. 2015} for further detail). The complete methodology is described in the online supplement.

From this phylogeny, we extracted 236 phylogenetically independent taxon pairs (Table \ref{tab:pair_div}). A fully resolved, bifurcating four-taxon phylogeny can have two basic topologies: \(\tt{((A,B),(C,D))}\) or \(\tt{((A,B),C),D))}\). Taxon pairs include all comparisons of \(\tt{A}\) with \(\tt{B}\) and \(\tt{C}\) with \(\tt{D}\) in each four-taxon clade. We extracted pairs using the \texttt{extract\_sisters()} function in R package \textbf{diverge} version 2.0.4 (\protect\hyperlink{ref-anderson_diverge_2021}{Anderson and Weir 2021}) and custom scripts (see source code). Taxon pairs are the most closely related pairs in our data set, but they are mostly not sister taxa in the sense of being the two most closely related taxa in the tree of life. For each pair we calculated phylogenetically independent contrasts (\protect\hyperlink{ref-felsenstein_phylogenies_1985}{Felsenstein 1985}) as the difference in the log\(_{10}\)-transformed trait value (see \protect\hyperlink{ref-beaulieu_genome_2008}{Beaulieu et al. 2008} for a similar approach). Contrasts are denoted as \(\Delta \text{log(trait)}\). We log-transformed traits for normality because like many morphological and anatomical traits they are strongly right-skewed. Log-transformation also helps compare density and length, which are measured on different scales, because log-transformed values quantify proportional rather than absolute divergence.

\hypertarget{parameter-estimation}{%
\subsection{Parameter estimation}\label{parameter-estimation}}

All hypotheses make predictions about trait (co)variance matrices or parameters derived from them (see the online supplement and subsections below). Within and among species covariation is a hallmark of developmental integration (\protect\hyperlink{ref-armbruster_multilevel_1988}{Armbruster 1988}), but other evolutionary processes also lead to covariance. Distinguishing between them requires deriving predictions and testing whether observed covariance is consistent with one hypothesis or another. We estimated the \(4 \times 4\) covariance matrix of phylogenetically independent contrasts between log-transformed values of \(\Delta \text{log}(D_\mathrm{ab})\), \(\Delta \text{log}(D_\mathrm{ad})\), \(\Delta \text{log}(L_\mathrm{ab})\), and \(\Delta \text{log}(L_\mathrm{ad})\) using a distributional multiresponse robust Bayesian approach. See Table \ref{tab:traits} for variable definitions. We denote variances as \(\text{Var}[\Delta \text{log(trait)}]\) and covariances as \(\text{Cov}[\Delta \text{log(trait}_1),\Delta \text{log(trait}_2)]\). We used a multivariate \(t\)-distribution rather than a Normal distribution because estimates using the former are more robust to exceptional trait values (\protect\hyperlink{ref-lange_robust_1989}{Lange, Little, and Taylor 1989}). We also estimated whether the variance in trait divergence increases with time. Under many trait evolution models (e.g.~Brownian motion), interspecific variance increases through time. To account for this, we included time since taxon-pair divergence as an explanatory variable affecting the trait covariance matrix.

For the packing limit hypothesis, we tested whether the variance in stomatal trait divergence, \(\textrm{Var}[\Delta~\textrm{log(trait)}]\), decreases as stomatal allocation increases. The fraction of epidermal area allocated to stomata (\(f_S\)) is the product of stomatal density and area occupied by a stomatal apparatus. Because guard cell shape is similar in most plant lineages except grasses, the area can be well approximated from guard cell length as \(A = j L ^ 2\) where \(j = 0.5\) for most species with kidney-shaped guard cells and \(j = 0.125\) for grasses with dumbbell-shaped guard cells (\protect\hyperlink{ref-sack_developmental_2016}{Sack and Buckley 2016}). For each contrast, we calculated the average \(f_S\) on each surface between those two taxa for use as our explanatory variable. The statistical model allowed the effect of \(f_S\) on \(\textrm{Var}[\Delta~\textrm{log(trait)}]\) to vary between traits and leaf surfaces. We also included time since taxon-pair divergence as an explanatory variable and used a multivariate \(t\)-distribution as described above.

We fit all models in Stan 2.29 (\protect\hyperlink{ref-stan_development_team_stan_2022}{Stan Development Team 2022}) using the R packages \textbf{brms} version 2.17.0 (\protect\hyperlink{ref-burkner_brms_2017}{Bürkner 2017}, \protect\hyperlink{ref-burkner_advanced_2018}{2018}) with a \textbf{cmdstanr} version 0.5.2 backend (\protect\hyperlink{ref-gabry_cmdstanr_2022}{Gabry and Češnovar 2022}). It ran on 2 parallel chains for 1000 warm-up iterations and 1000 sampling iterations. All parameters converged (\(\hat{R} \approx 1\)) and the effective sample size from the posterior exceeded 1000 (\protect\hyperlink{ref-vehtari_rank-normalization_2021}{Vehtari et al. 2021}). We used the posterior median for point estimates and calculated uncertainty with the 95\% highest posterior density (HPD) interval from the posterior distribution.

\hypertarget{hypothesis-testing}{%
\subsection{Hypothesis testing}\label{hypothesis-testing}}

\hypertarget{does-divergence-slow-as-epidermal-space-fills}{%
\subsubsection{Does divergence slow as epidermal space fills?}\label{does-divergence-slow-as-epidermal-space-fills}}

We tested the packing limit hypothesis by estimating whether the fraction of epidermal area allocated to stomata (\(f_S\)) effects \(\textrm{Var}[\Delta~\textrm{log(trait)}]\) for each trait and leaf surface. If there is an upper bound on \(f_S\), we predict the effect of \(f_S\) on \(\textrm{Var}[\Delta~\textrm{log(trait)}]\) will be \(<0\). Specifically, the 95\% HPD intervals should not include 0. Further, the coefficient should be the same on each surface, so the 95\% HPD intervals for difference should encompass 0.

\hypertarget{is-size-density-scaling-the-same-on-both-leaf-surfaces}{%
\subsubsection{Is size-density scaling the same on both leaf surfaces?}\label{is-size-density-scaling-the-same-on-both-leaf-surfaces}}

We tested whether the covariance between divergence in stomatal length and stomatal density on each leaf surface is the same. If size and density are developmentally integrated, we predict the covariance matrices will not be significantly different. Specifically, the 95\% HPD intervals of the difference in covariance parameters should not include 0 if:

\begin{gather}\label{eq:prediction1}
\text{Var}[\Delta \text{log}(D_\text{ab})] \ne \text{Var}[\Delta \text{log}(D_\text{ad})] \\
\text{Var}[\Delta \text{log}(L_\text{ab})] \ne \text{Var}[\Delta \text{log}(L_\text{ad})] \\
\text{Cov}[\Delta \text{log}(L_\text{ab}), \Delta \text{log}(D_\text{ab})] \ne \text{Cov}[\Delta \text{log}(L_\text{ad}), \Delta \text{log}(D_\text{ad})]
\end{gather}

\hypertarget{do-abaxial-and-adaxial-stomatal-traits-evolve-isometrically}{%
\subsubsection{Do abaxial and adaxial stomatal traits evolve isometrically?}\label{do-abaxial-and-adaxial-stomatal-traits-evolve-isometrically}}

If stomatal traits on each surface are developmentally integrated then divergence in the trait on one surface should result in a 1:1 (isometric) change in the trait on the other surface. Furthermore, there should be relatively little variation away from a 1:1 relationship. Conversely, if traits can evolve independently then the change in the trait on one surface should be uncorrelated with changes on the other. We tested for isometry by estimating the standardized major axis (SMA) slope of divergence in the abaxial trait against divergence in the adaxial trait for both stomatal length and stomatal density. If change on each surface is isometric, then the HPD intervals for the slope should include 1. We used the coefficient of determination, \(r^2\), to quantify the strength of integration, where a value of 1 is complete integration and a value of 0 is complete disintegration.

\hypertarget{results}{%
\section{Results}\label{results}}

\hypertarget{theory-from-developmental-integration-to-disintegration}{%
\subsection{Theory: from developmental integration to disintegration}\label{theory-from-developmental-integration-to-disintegration}}

We asked how divergence in stomatal length and density would covary if the developmental function were constrained and compared it to their divergence when the developmental function can evolve. When the developmental function is constrained this means that allocation to guard cell meristemoids during asymmetric division and stomatal index are fixed (see the online supplement for mathematical description). Under these assumptions, divergence in stomatal length and density is mediated entirely by divergence in meristematic cell volume and expansion prior to differentiation. Developmental integration is strong because divergence in density is perfectly negatively correlated with divergence in size. In contrast, stomatal length and density can diverge independently when the developmental function is not fixed. Divergence in asymmetric cell division affects stomatal size independently of density; divergence in stomatal index affects stomatal density independently of size. Divergence in the developmental function causes developmental disintegration because stomatal density and size can diverge independently. Developmental integration is minimal when asymmetric cell division and/or stomatal index diverge more than meristematic cell volume and expansion. The three main conclusions are that 1) developmental constraint leads to developmental integration; 2) different (co)variance in divergence of stomatal length and density on each surface implies the developmental function is not fixed; and 3) divergence in different components of the developmental function affect stomatal length and density differently. See the online supplement and Table \ref{tab:predictions} for more a complete derivation and detailed predictions.

\hypertarget{empirical-results}{%
\subsection{Empirical results}\label{empirical-results}}

The variance in trait divergence decreases as the fraction of epidermal area allocated to stomata, \(f_S\), increases (Figs. \ref{fig:h3}, \ref{fig:fs-sigma}). The effect of \(f_S\) was strongest for adaxial stomatal density (\(D_\text{ad}\)) and 95\% HPD intervals did not overlap 0 for 3 of 4 comparisons (Table \ref{tab:h3output}). Variance in divergence for \(D_\text{ad}\) declined more rapidly with \(f_S\) than that for abaxial stomatal density (\(D_\text{ab}\); difference and 95\% HPD interval in slope, log-link scale: -10.7 {[}-17.2,-4.2{]}). Variance in length divergence declined similarly with \(f_S\) on both surfaces (difference and 95\% HPD interval in slope, log-link scale: -2.8 {[}-9.6,4{]}).

Stomatal length negatively covaries with stomatal density similarly on both surfaces, but on the adaxial surface there are many more taxa that have low stomatal density and small size compared to the abaxial surface (Fig. \ref{fig:h1_raw}). In principle, this pattern could arise either because size-density covariance differs or the variance in adaxial stomatal density increases faster than that for abaxial stomatal density. The interspecific variance increases with time since divergence for all traits (Table \ref{tab:h12output}). For consistency, we therefore report estimates conditional on time since divergence set to 0. Across pairs, we estimate that the covariance between size and density is similar. The median estimate is \(\text{Cov}[\Delta \text{log}(L_\text{ad}), \Delta \text{log}(D_\text{ad})] - \text{Cov}[\Delta \text{log}(L_\text{ab}), \Delta \text{log}(D_\text{ab})] = 3.18 \times 10^{-4}\), but 0 is within the range of uncertainty (95\% HPD interval \([-3.85 \times 10^{-3},4.26 \times 10^{-3}]\)). However the variance in adaxial stomatal density is significantly greater than the abaxial stomatal density {[}Fig. \ref{fig:h1}{]}). We estimate \(\text{Var}[\Delta \text{log}(D_\text{ad})]\) is \(4.00 \times 10^{-2}\) (95\% HPD interval \([1.42 \times 10^{-2},7.07 \times 10^{-2}]\)) greater than \(\text{Var}[\Delta \text{log}(D_\text{ab})]\). The variance in stomatal length was similar for both surfaces, with an estimate of \(-4.54 \times 10^{-4}\) (95\% HPD interval \([-1.67 \times 10^{-3},7.45 \times 10^{-4}]\)).

We analyzed a smaller set of 79 contrasts with data on both genome size and stomatal anatomy. Consistent with previous studies (\protect\hyperlink{ref-beaulieu_genome_2008}{Beaulieu et al. 2008}; \protect\hyperlink{ref-jordan_environmental_2015}{Jordan et al. 2015}; \protect\hyperlink{ref-simonin_genome_2018}{Simonin and Roddy 2018}), increased genome size was associated with increased stomatal length on both surfaces (Fig. \ref{fig:genome}). The association between genome size and stomatal density was negative, as expected, but weaker. Only the slope for adaxial stomatal density was significantly less than 0 (Fig. \ref{fig:genome}).

Stomatal density on each surface is less integrated than stomatal length. The relationship between stomatal density on each leaf surface is visually more variable than that for stomatal length (Fig. \ref{fig:h2_raw}). This pattern occurs because the slope and strength of integration for stomatal density on each surface is much weaker than that for stomatal length. The SMA slope between \(\Delta \text{log}(D_\text{ad})\) and \(\Delta \text{log}(D_\text{ab})\) is less than 1 (estimated slope \(= 0.742\), 95\% HPD interval \([0.619,0.883]\)) and the strength of association is weakly positive (estimated \(r^2 = 0.113\), 95\% HPD interval \([0.0431,0.205]\); Fig. \ref{fig:h2}). In contrast, the relationship between \(\Delta \text{log}(L_\text{ad})\) and \(\Delta \text{log}(L_\text{ab})\) is isometric (estimated slope \(= 1.03\), 95\% HPD interval \([0.955,1.12]\)) and strongly positive (estimated \(r^2 = 0.762\), 95\% HPD interval \([0.691,0.82]\); Fig. \ref{fig:h2}).

\hypertarget{discussion}{%
\section{Discussion}\label{discussion}}

Functional and developmental constraints may slow adaptation by preventing traits from evolving independently towards a multivariate phenotypic optimum. An alternative view is that phenotypic evolution is limited primarily because most phenotypes are maladaptive and selected against, sometimes referred to as selective constraint. In this study, we took advantage of the fact that amphistomatous leaves produce stomata on both abaxial (usually lower) and adaxial (usually upper) surfaces. Packing limits, a functional constraint, and developmental integration should result in similar (co)variance in divergence of stomatal traits on each surface (see the online supplement), whereas differing selective pressures for each surface would lead to different patterns of divergence. Although some patterns of trait (co)variance are compatible with packing limits or developmental integration (see below), independent evolution of stomatal density on each surface across flowering plants is notably inconsistent with these constraints. We therefore conclude that selection against generally maladaptive trait combinations plays a major role in the evolution of stomata and possibly other ecologically important traits. It should not be surprising that stomatal traits respond to natural selection, but it is nevertheless plausible that some stomatal trait combinations that would have high fitness are precluded by nonadaptive constraints. Our results challenge these nonadaptive explanations, especially for stomatal density.

Neither packing limits nor developmental integration were sufficient to explain two major patterns of divergence in stomatal traits (Fig. \ref{fig:concepts}). Although evolutionary divergence slowed as the allocation to stomata \(f_S\) increased, the effect of \(f_S\) was different on each surface (Fig. \ref{fig:h3} and \ref{fig:fs-sigma}; Table \ref{tab:h3output}). The contrasting pattern of divergence on each surface is inconsistent with a common packing limit and suggests instead that a different range of adaptive stomatal trait combinations on the lower and upper leaf surface. Contrary to the developmental integration hypotheses, the greater variance in stomatal density compared to length on the adaxial surface indicates that density is more labile on this surface, though traits on each surface are not completely decoupled (Fig. \ref{fig:h1}, \ref{fig:h2}; Table \ref{tab:h12output}). Consistent with the developmental integration hypotheses, divergence in stomatal length on each surface evolves isometrically at the same rate, suggesting that guard cell dimensions may not be able to evolve independently on each surface (Fig. \ref{fig:h2}). The evolutionary lability of stomatal density, despite constraints on size, show that inverse size-density scaling and bimodal stomatal ratio cannot be attributed entirely to developmental integration. Combinations of small stomata and low density that are not found on the abaxial surface are found on the adaxial surface, indicating that these rare trait combinations are developmentally accessible.

While phylogenetic comparisons usually cannot prove that phenotypic variation is adaptive, ruling out alternative hypotheses as the sole explanation is an important step toward quantifying the relative importance of adaptation in macroevolution (\protect\hyperlink{ref-mcghee_theoretical_1999}{McGhee 1999}; \protect\hyperlink{ref-olson_plant_2019}{Olson 2019}). Establishing that traits can evolve quasi-independently is necessary but not sufficient to show that selection is the primary process shaping phenotypic evolution. Packing limits and developmental integration may bias phenotypic evolution, even if they do not preclude certain stomatal trait combinations. Therefore, future research will need to combine stomatal developmental (dis)integration with biophysical models of how stomatal anatomy would vary adaptively (\protect\hyperlink{ref-olson_how_2015}{Olson and Arroyo-Santos 2015}). Although these questions and approaches apply to any phenotype, stomata will be a useful trait because of their ecological significance and broad application to most land plants.

\hypertarget{do-packing-limits-and-developmental-integration-lead-to-inverse-size-density-scaling}{%
\subsection{Do packing limits and developmental integration lead to inverse size-density scaling?}\label{do-packing-limits-and-developmental-integration-lead-to-inverse-size-density-scaling}}

Stomata cannot occupy more than the entire leaf surface, but realistically there is probably an upper packing limit below this hard bound. If this packing limit drives inverse size-density scaling, we should observe that divergence in stomatal size and density decrease as this limit is approached. Near the limit, large changes that reduce \(f_S\) are possible, but changes that increase \(f_S\) must be small so as not to exceed the limit. Furthermore, the same packing limit should apply to both ab- and adaxial leaf surfaces. Although we observe that divergence decreases with \(f_S\), the relationship is not the same on both surfaces (Fig. \ref{fig:h3} and \ref{fig:fs-sigma}; Table \ref{tab:h3output}). This implies that other factors constrain stomatal size and density before they approach a packing limit.

If meristematic cell volume and expansion integrate stomatal size and density (\protect\hyperlink{ref-brodribb_unified_2013}{Brodribb, Jordan, and Carpenter 2013}), then we predicted inverse size-density scaling would evolve with the same (co)variance for both ab- and adaxial leaf surfaces (see the online supplement). Contrary to this prediction, there are many combinations of stomatal density (\(D_i\)) and length (\(L_i\)) found on adaxial leaf surfaces that are absent from abaxial leaf surfaces (Fig. \ref{fig:h1_raw}). In principle, the different relationship between traits on each surface could be caused by different evolutionary variance in stomatal density (\(\text{Var}[\Delta \text{log}(D_\text{ab})] \ne \text{Var}[\Delta \text{log}(D_\text{ad})]\)) and/or covariance (\(\text{Cov}[\Delta \text{log}(L_\text{ab}), \Delta \text{log}(D_\text{ab})] \ne \text{Cov}[\Delta \text{log}(L_\text{ad}), \Delta \text{log}(D_\text{ad})]\)) on each surface. However, the covariance relationship between density and length is similar on each surface, whereas the evolutionary variance in adaxial stomatal density is significantly higher than that for abaxial density (\(\text{Var}[\Delta \text{log}(D_\text{ab})] < \text{Var}[\Delta \text{log}(D_\text{ad})]\); Fig. \ref{fig:h1}). Given that the average stomatal length is usually about the same on each surface (see below), these results imply that plants can often evolve stomatal densities on each surface without a concomitant change in size. Based on our theoretical analysis, we interpret these results to mean that cell divisions affecting stomatal index are less evolutionarily constrained than the asymmetric cell division preceding the guard cell meristemoid (Fig. \ref{fig:developmental-integration}; Table \ref{tab:predictions})

The disintegration of stomatal size and density on adaxial leaf surfaces implies that the inverse size-density scaling on abaxial surfaces (\protect\hyperlink{ref-weiss_untersuchungen_1865}{Weiss 1865}; \protect\hyperlink{ref-franks_maximum_2009}{Franks and Beerling 2009}; \protect\hyperlink{ref-de_boer_optimal_2016}{de Boer et al. 2016}; \protect\hyperlink{ref-sack_developmental_2016}{Sack and Buckley 2016}; \protect\hyperlink{ref-liu_scaling_2021}{Liu et al. 2021}) is not a developmental \emph{fait accompli}. The lability of \(D_\text{ad}\) may explain why there is so much putatively adaptive variation in the trait along light gradients (\protect\hyperlink{ref-muir_light_2018}{Muir 2018}) and in coordination with other anatomical traits that vary among precipitation habitats (\protect\hyperlink{ref-pathare_increased_2020}{Pathare, Koteyeva, and Cousins 2020}). There may appear to be a tension between our results and recent findings that genome size, which is strongly correlated with meristematic cell volume (\protect\hyperlink{ref-simova_geometrical_2012}{Šímová and Herben 2012}), correlates strongly with mature guard cell size as well as the size and packing density of mesophyll cells (\protect\hyperlink{ref-roddy_scaling_2020}{Roddy et al. 2020}; \protect\hyperlink{ref-theroux-rancourt_maximum_2021}{Théroux-Rancourt et al. 2021}). However, most plant species are far from their minimum cell size as determined by genome size {[}Roddy et al. (\protect\hyperlink{ref-roddy_scaling_2020}{2020}); Fig. \ref{fig:h1_raw}{]}. Genome size explains 31-54\% of the variation in stomatal density across the major groups of terrestrial plants (\protect\hyperlink{ref-simonin_genome_2018}{Simonin and Roddy 2018}) but there is huge variation in stomatal density and stomatal length in angiosperms with rather similar genome size (\emph{c.f.} Fig. 2 in \protect\hyperlink{ref-simonin_genome_2018}{Simonin and Roddy 2018}). Genome size, a proxy for meristematic cell volume, is more strongly related to stomatal size than density (Fig. \ref{fig:genome}). Yet the decoupling of size and density on the adaxial surface suggests that meristematic cell volume is probably not a strong constraint on the final size of epidermal pavement and guard cells because of different division and expansion rates after the asymmetric cell division stage. A possible resolution is that meristematic cell volume limits the range of variation in species with exceptionally large genome, but most species can modify stomatal size and density independently of each other to optimize photosynthesis (\protect\hyperlink{ref-jordan_environmental_2015}{Jordan et al. 2015}; \protect\hyperlink{ref-simonin_genome_2018}{Simonin and Roddy 2018}; \protect\hyperlink{ref-roddy_scaling_2020}{Roddy et al. 2020}; \protect\hyperlink{ref-theroux-rancourt_maximum_2021}{Théroux-Rancourt et al. 2021}).

\hypertarget{does-developmental-integration-lead-to-bimodal-stomatal-ratio}{%
\subsection{Does developmental integration lead to bimodal stomatal ratio?}\label{does-developmental-integration-lead-to-bimodal-stomatal-ratio}}

We predicted that if abaxial and adaxial stomata are developmentally integrated then we should observe a strong, isometric relationship between trait divergence on each surface. Consistent with this prediction, divergence in stomatal length on each surface is isometric (SMA slope \(=\) 1.03) and strongly associated (\(r^2 =\) 0.762; Fig. \ref{fig:h2}). In contrast, divergence in stomatal density on each surface was not isometric (SMA slope \(=\) 0.742) and much less integrated (\(r^2 =\) 0.113; Fig. \ref{fig:h2}). Since average stomatal density on each surface can evolve quasi-independently, a wide variety of stomatal ratios are developmentally possible. The stomatal developmental function is not constrained to be identical on each surface. If natural selection favored an intermediate stomatal ratio between modes, developmental processes should not constrain the expression of genetic variation to achieve it. This supports the hypothesis that the bimodal stomatal ratio pattern (\protect\hyperlink{ref-muir_making_2015}{Muir 2015}) arises because natural selection rarely favors intermediate trait values, not because intermediate values are developmentally inaccessible.

\hypertarget{limitations-and-future-research}{%
\subsection{Limitations and future research}\label{limitations-and-future-research}}

The ability of adaxial stomatal density to evolve independently of stomatal size and abaxial stomatal density is not consistent with packing limits or developmental integration as the primary cause leading to inverse size-density scaling or bimodal stomatal ratio. However, there are two major limitations of this study that should be addressed in future work. First, while \(D_\text{ab}\) can diverge independently of other stomatal traits globally, we cannot rule out that developmental integration is important in some lineages. Developmental constraints are often localized to particular clades but not universal (\protect\hyperlink{ref-maynard_smith_developmental_1985}{Maynard Smith et al. 1985}). For example, Berg's rule observes that vegetative and floral traits are often developmentally integrated, but integration can be broken when selection favors flowers for specialized pollination (\protect\hyperlink{ref-berg_general_1959}{Berg 1959}, \protect\hyperlink{ref-berg_ecological_1960}{1960}; \protect\hyperlink{ref-conner_raissa_2014}{Conner and Lande 2014}). Other traits evince developmental modularity, such as the independent evolution leaf and petal venation (\protect\hyperlink{ref-roddy_uncorrelated_2013}{Roddy et al. 2013}). Analogously, developmental integration between stomatal anatomical traits could evolve in some lineages, due to selection or other evolutionary forces, but become less integrated in other lineages. For example, \(D_\text{ab}\) and \(D_\text{ad}\) are positively genetically correlated in \emph{Oryza} (\protect\hyperlink{ref-ishimaru_identification_2001}{Ishimaru et al. 2001}; \protect\hyperlink{ref-rae_elucidating_2006}{Rae et al. 2006}), suggesting developmental integration may contribute to low variation in stomatal ratio between species of this genus (\protect\hyperlink{ref-giuliani_coordination_2013}{Giuliani et al. 2013}). A second major limitation is that covariation in traits like stomatal length, which appear to be developmentally integrated on each surface, could be caused by other processes. For example, since stomatal size affects the speed and mechanics of stomatal closure (\protect\hyperlink{ref-drake_smaller_2013}{Drake, Froend, and Franks 2013}; \protect\hyperlink{ref-harrison_influence_2020}{Harrison et al. 2020}; but see \protect\hyperlink{ref-roddy_scaling_2020}{Roddy et al. 2020}), there may be strong selection for similar stomatal size throughout the leaf to harmonize rates of stomatal closure. Coordination between epidermal and mesophyll development may also constrain how independently stomatal traits on each surface can evolve (\protect\hyperlink{ref-dow_disruption_2017}{Dow, Berry, and Bergmann 2017}; \protect\hyperlink{ref-lundgren_mesophyll_2019}{Lundgren et al. 2019}; \protect\hyperlink{ref-theroux-rancourt_maximum_2021}{Théroux-Rancourt et al. 2021}; \protect\hyperlink{ref-borsuk_structural_2022}{Borsuk et al. 2022}).

Future research should identify the mechanistic basis of developmental disintegration between \(D_\text{ab}\) and \(D_\text{ad}\). Multiple reviews of stomatal development conclude that stomatal traits are independently controlled on each surface (\protect\hyperlink{ref-lake_longdistance_2002}{Lake, Woodward, and Quick 2002}; \protect\hyperlink{ref-bergmann_stomatal_2007}{Bergmann and Sack 2007}), but we do not know much about linkage between ab-adaxial polarity and stomatal development (\protect\hyperlink{ref-kidner_signaling_2010}{Kidner and Timmermans 2010}; \protect\hyperlink{ref-pillitteri_mechanisms_2012}{Pillitteri and Torii 2012}). Systems that have natural variation in stomatal ratio should allow us to study how developmental disintegration evolves. Quantitative genetic studies in \emph{Brassica oleracea} L., \emph{Oryza sativa} L., \emph{Populus trichocarpa} Torr. \& A. Gray ex Hook., \emph{Populus} interspecific crosses, and \emph{Solanum} interspecific crosses, typically find partial independence of \(D_\text{ab}\) and \(D_\text{ad}\); some loci affect both traits, but some loci only affect density on one surface and/or genetic correlations are weak (\protect\hyperlink{ref-ishimaru_identification_2001}{Ishimaru et al. 2001}; \protect\hyperlink{ref-ferris_leaf_2002}{Ferris et al. 2002}; \protect\hyperlink{ref-hall_relationships_2005}{Hall et al. 2005}; \protect\hyperlink{ref-rae_elucidating_2006}{Rae et al. 2006}; \protect\hyperlink{ref-laza_quantitative_2010}{Laza et al. 2010}; \protect\hyperlink{ref-chitwood_quantitative_2013}{Chitwood et al. 2013}; \protect\hyperlink{ref-mckown_association_2014}{McKown et al. 2014}; \protect\hyperlink{ref-muir_quantitative_2014}{Muir, Pease, and Moyle 2014}; \protect\hyperlink{ref-porth_evolutionary_2015}{Porth et al. 2015}; \protect\hyperlink{ref-fetter_growthdefense_2021}{Fetter, Nelson, and Keller 2021}). For example, \emph{Populus trichocarpa} populations have putatively adaptive genetic variation in \(D_\text{ad}\). Populations are more amphistomatous at Northern latitudes with shorter growing seasons that may select for faster carbon assimilation (\protect\hyperlink{ref-mckown_association_2014}{McKown et al. 2014}; \protect\hyperlink{ref-kaluthota_higher_2015}{Kaluthota et al. 2015}; \protect\hyperlink{ref-porth_evolutionary_2015}{Porth et al. 2015}). Genetic variation in key stomatal development transcription factors is associated with latitudinal variation in \(D_\text{ad}\), which should help reveal mechanistic basis of developmental disintegration between surfaces (\protect\hyperlink{ref-mckown_role_2019}{McKown et al. 2019}).

\hypertarget{acknowledgements}{%
\section{Acknowledgements}\label{acknowledgements}}

EJ Edwards proposed the idea of how developmental integration could explain bimodal stomatal ratio. Students in the University of Hawaii BIOL 470: Evolutionary Biology course helped enter data from the literature. We thank Matt Pennell and Jacob Watts for comments on earlier versions of this manuscript. Jen Lau, Risa Sargent, Adam Roddy, and an anonymous reviewer provided helpful comments on an earlier version of this manuscript. CDM was supported by an Evo-Devo-Eco Network (EDEN) research exchange (NSF IOS \#0955517). The research was supported by project AGL2013-42364-R (Plan Nacional, Spain) and UIB Grant 15/2105 awarded to JG. We acknowledge Miquel Truyols and collaborators of the UIB Experimental Field and Greenhouses for their technical support. This is publication \#178 from the School of Life Sciences, University of Hawaii at Mānoa.

\hypertarget{statement-of-authorship}{%
\section{Statement of Authorship}\label{statement-of-authorship}}

CDM led conceptualization, data analysis, and writing the original draft. All authors collected data and assisted with review and editing.

\hypertarget{data-avaibility}{%
\section{Data avaibility}\label{data-avaibility}}

The final data set and phylogeny used in the analysis are included in the Online Supplement. The raw anatomical data and source code are deposited in the Dryad Digital Repository: \url{https://doi.org/10.5061/dryad.0rxwdbs42} (\protect\hyperlink{ref-muir_data_2022}{Muir et al. 2022}).

\clearpage

\hypertarget{tables}{%
\section{Tables}\label{tables}}

\begin{longtable}[]{@{}
  >{\centering\arraybackslash}p{(\columnwidth - 4\tabcolsep) * \real{0.1889}}
  >{\centering\arraybackslash}p{(\columnwidth - 4\tabcolsep) * \real{0.5444}}
  >{\centering\arraybackslash}p{(\columnwidth - 4\tabcolsep) * \real{0.2667}}@{}}
\caption{\label{tab:traits}Stomatal anatomical traits with mathemtical symbol, description, and scientific units.}\tabularnewline
\toprule()
\begin{minipage}[b]{\linewidth}\centering
Symbol
\end{minipage} & \begin{minipage}[b]{\linewidth}\centering
Definition
\end{minipage} & \begin{minipage}[b]{\linewidth}\centering
Units
\end{minipage} \\
\midrule()
\endfirsthead
\toprule()
\begin{minipage}[b]{\linewidth}\centering
Symbol
\end{minipage} & \begin{minipage}[b]{\linewidth}\centering
Definition
\end{minipage} & \begin{minipage}[b]{\linewidth}\centering
Units
\end{minipage} \\
\midrule()
\endhead
\(D_\mathrm{ab}\) & Stomatal density on abaxial (lower) surface & \(\text{pores mm}^{-2}\) \\
\(D_\mathrm{ad}\) & Stomatal density on adaxial (upper) surface & \(\text{pores mm}^{-2}\) \\
\(f_\mathrm{S}\) & Fraction of epidermal area allocated to stomata & unitless \\
\(L_\mathrm{ab}\) & Guard cell length on abaxial (lower) surface & \(\mu\)m \\
\(L_\mathrm{ad}\) & Guard cell length on adaxial (upper) surface & \(\mu\)m \\
\bottomrule()
\end{longtable}

\clearpage

\hypertarget{figure-legends}{%
\section{Figure legends}\label{figure-legends}}

\begin{figure}[ht]
% \includegraphics[width=\textwidth]{../figures/concepts.pdf}
\caption{Two salient features of stomatal anatomy in flowering plants are the (a) inverse relationship between stomatal size and density and (b) the bimodal distribution of stomatal ratio. At broad phylogenetic scales, species with smaller stomata on their leaves ($x$-axis, log-scale) tend to have greater stomatal density  ($x$-axis, log-scale), but there is a lot of variation about the overall trend indicated by the grey ellipse. Hypostomatous leaves (stomatal ratio = 0) are more common than amphistomatoues leaves, but within amphistomatous leaves, the density of stomata on each surface tends to be similar (stomatal ratio $\approx$ 0.5), which we refer to as bimodal stomatal ratio.}
\label{fig:concepts}
\end{figure}

\begin{figure}[ht]
% \includegraphics[width=\textwidth]{../figures/h3.pdf}
\caption{Evolutionary divergence slows down as epidermal space fills up. This pattern is consistent with functional constraints (packing limits) constraining evolution, but only when stomata occupy a large fraction of epidermal area. The shaded area in each facet indicates the estimate of where 95\% of the 236 phylogenetically independent contrasts fall as a function of the fraction of epidermal area allocated to stomata per surface. Each point is the absolute value of  $\Delta~\textrm{log(trait)}$ for stomatal density (left facets) or length (right facets) on the adaxial (upper facets) and abaxial (lower facets) surface. The fraction of epidermal area allocated to stomata is the average value per surface between the two taxa in each contrast. Divergence in anatomical traits is more variable when stomata occupy a smaller area, especially for adaxial stomatal density (upper left facet). See Table \ref{tab:h3output} for all parameter estimates and confidence intervals.}
\label{fig:h3}
\end{figure}

\begin{figure}[ht]
% \includegraphics[width=\textwidth]{../figures/h1-raw.pdf}
\caption{Inverse size-density scaling differs between the surfaces of amphistomatous leaf, indicating different phenotypic constraints. The panels show the relationship between stomatal length ($x$-axis) and stomatal density ($y$-axis) on a log-log scale for values measured on the abaxial leaf surface (left) and the adaxial leaf surface (right) across 638 taxa.}
\label{fig:h1_raw}
\end{figure}

\begin{figure}[ht]
% \includegraphics[width=\textwidth]{../figures/h1.pdf}
\caption{(Previous page.) Greater overall phenotypic constraint on abaxial stomatal anatomy is evinced by more variable evolutionary divergence in adaxial stomatal density, but covariance between density and length is similar on both surfaces. (a) Data from 236 phylogenetically independent contrasts of change in log(stomatal length) ($x$-axis) and log(stomatal density) ($y$-axis) for abaxial (left panel) and adaxial (right panel) leaf surfaces. Each contrast is shown by black points and every contrast appears on both panels. Grey ellipses are the model-estimated 95\% covariance ellipses. The negative covariance is similar for both surfaces but the breadth in the $y$-direction is larger for adaxial traits, indicating greater evolutionary divergence in log(stomatal density). (b) Parameter estimates (points), 66\% (thick lines), and 95\% HPD intervals for estimates of trait (co)variance. Grey points and lines represent ab- and adaxial values; black points and lines represent the estimated difference in (co)variance between surfaces. Only the variance for stomatal density (middle panel) is significantly greater for the adaxial surface (95\% HPD interval does not overlap the dashed line at 0). Reported parameter estimates are conditioned on zero time since divergence between taxa (see \protect\hyperlink{results}{Results}).}
  \label{fig:h1}
\end{figure}

\begin{figure}[ht]
% \includegraphics[width=\textwidth]{../figures/h2-raw.pdf}
\caption{Stomatal density on each surface is less constrained (lower correlation) than stomatal length (higher correlation). Relationship between stomatal density and length on each leaf surface in a synthesis of amphistomatous leaf traits across 638 taxa. The panels show the relationship between the abaxial trait value ($x$-axis) and the adaxial trait value ($y$-axis) on a log-log scale for stomatal density (left) and stomatal length (right). The dashed line in across the middle is the 1:1 line for reference.}
\label{fig:h2_raw}
\end{figure}

\begin{figure}[ht]
% \includegraphics[width=\textwidth]{../figures/h2.pdf}
\caption{Developmental integration in stomatal length is much stronger than stomatal density between the surfaces of amphistomatous leaves (a) Data from 236 phylogenetically independent contrasts of change in the abaxial trait value ($x$-axis) against change in the adaxial trait value ($y$-axis) for log(stomatal density) (left panel) and log(stomatal length) (right panel). Each contrast is shown by black points and every contrast appears on both panels. Dashed grey lines are 1:1 lines for reference. Solid grey lines and ribbon the fitted SMA slope and 95\% HPD interval. (b) The SMA slope (left panel) is significantly less than 1 (isometry, top dashed line) for density but very close to isometric for length. The coefficient of determination ($r^2$, right panel) is also much greater for length than density. The points are parameter estimates with 66\% (thick lines) and 95\% HPD intervals. Reported parameter estimates are conditioned on zero time since divergence between taxa (see \protect\hyperlink{results}{Results}).}
\label{fig:h2}
\end{figure}

\clearpage

\renewcommand\thefigure{S\arabic{figure}}    
\renewcommand\thetable{S\arabic{table}}    
\renewcommand\theequation{S\arabic{equation}}    
\setcounter{figure}{0}    
\setcounter{table}{0}    
\setcounter{equation}{0}

\begin{figure}[ht]
  \captionsetup{labelformat=empty}
  \caption{}
  \label{fig:fs-sigma}
\end{figure}

\begin{figure}[ht]
  \captionsetup{labelformat=empty}
  \caption{}
  \label{fig:genome}
\end{figure}

\begin{figure}[ht]
  \captionsetup{labelformat=empty}
  \caption{}
  \label{fig:developmental-integration}
\end{figure}

\begin{figure}[ht]
  \captionsetup{labelformat=empty}
  \caption{}
  \label{fig:check-approximation}
\end{figure}

\begin{table}
  \captionsetup{labelformat=empty}
  \caption{}
  \label{tab:pair_div}
\end{table}

\begin{table}[ht]
  \captionsetup{labelformat=empty}
  \caption{}
  \label{tab:h3output}
\end{table}

\begin{table}[ht]
  \captionsetup{labelformat=empty}
  \caption{}
  \label{tab:h12output}
\end{table}

\begin{table}[ht]
  \captionsetup{labelformat=empty}
  \caption{}
  \label{tab:predictions}
\end{table}

\begin{table}[ht]
  \captionsetup{labelformat=empty}
  \caption{}
  \label{tab:sources}
\end{table}

\hypertarget{literature-cited}{%
\section*{Literature Cited}\label{literature-cited}}
\addcontentsline{toc}{section}{Literature Cited}

\hypertarget{refs}{}
\begin{CSLReferences}{1}{0}
\leavevmode\vadjust pre{\hypertarget{ref-anderson_diverge_2021}{}}%
Anderson, Sean A S, and Jason T Weir. 2021. \emph{Diverge: {Evolutionary} {Trait} {Divergence} {Between} {Sister} {Species} and {Other} {Paired} {Lineages}}. \url{https://CRAN.R-project.org/package=diverge}.

\leavevmode\vadjust pre{\hypertarget{ref-armbruster_multilevel_1988}{}}%
Armbruster, W. Scott. 1988. {``Multilevel {Comparative} {Analysis} of the {Morphology}, {Function}, and {Evolution} of \emph{Dalechampia} {Blossoms}.''} \emph{Ecology} 69 (6): 1746--61. \url{https://doi.org/10.2307/1941153}.

\leavevmode\vadjust pre{\hypertarget{ref-armbruster_covariance_1999}{}}%
Armbruster, W. Scott, Vero´nica S. Di Stilio, John D. Tuxill, T. Christopher Flores, and Julie L. Vela´squez Runk. 1999. {``Covariance and Decoupling of Floral and Vegetative Traits in Nine {Neotropical} Plants: A Re‐evaluation of {Berg}'s Correlation‐pleiades Concept.''} \emph{American Journal of Botany} 86 (1): 39--55. \url{https://doi.org/10.2307/2656953}.

\leavevmode\vadjust pre{\hypertarget{ref-armbruster_integrated_2014}{}}%
Armbruster, W. Scott, Christophe Pélabon, Geir H. Bolstad, and Thomas F. Hansen. 2014. {``Integrated Phenotypes: Understanding Trait Covariation in Plants and Animals.''} \emph{Philosophical Transactions of the Royal Society B: Biological Sciences} 369 (1649): 20130245. \url{https://doi.org/10.1098/rstb.2013.0245}.

\leavevmode\vadjust pre{\hypertarget{ref-arnold_constraints_1992}{}}%
Arnold, Stevan J. 1992. {``Constraints on Phenotypic Evolution.''} \emph{The American Naturalist} 140: S85--107.

\leavevmode\vadjust pre{\hypertarget{ref-barrett_sexual_2013}{}}%
Barrett, Spencer C. H., and Josh Hough. 2013. {``Sexual Dimorphism in Flowering Plants.''} \emph{Journal of Experimental Botany} 64 (1): 67--82. \url{https://doi.org/10.1093/jxb/ers308}.

\leavevmode\vadjust pre{\hypertarget{ref-beaulieu_genome_2008}{}}%
Beaulieu, Jeremy M., Ilia J. Leitch, Sunil Patel, Arjun Pendharkar, and Charles A. Knight. 2008. {``Genome Size Is a Strong Predictor of Cell Size and Stomatal Density in Angiosperms.''} \emph{New Phytologist} 179 (4): 975--86. \url{https://doi.org/10.1111/j.1469-8137.2008.02528.x}.

\leavevmode\vadjust pre{\hypertarget{ref-berg_general_1959}{}}%
Berg, R. L. 1959. {``A {General} {Evolutionary} {Principle} {Underlying} the {Origin} of {Developmental} {Homeostasis}.''} \emph{The American Naturalist} 93 (869): 103--5. \url{https://doi.org/10.1086/282061}.

\leavevmode\vadjust pre{\hypertarget{ref-berg_ecological_1960}{}}%
---------. 1960. {``The {Ecological} {Significance} of {Correlation} {Pleiades}.''} \emph{Evolution} 14 (2): 171. \url{https://doi.org/10.2307/2405824}.

\leavevmode\vadjust pre{\hypertarget{ref-bergmann_stomatal_2007}{}}%
Bergmann, Dominique C., and Fred D. Sack. 2007. {``Stomatal {Development}.''} \emph{Annual Review of Plant Biology} 58 (1): 163--81. \url{https://doi.org/10.1146/annurev.arplant.58.032806.104023}.

\leavevmode\vadjust pre{\hypertarget{ref-berry_stomata:_2010}{}}%
Berry, Joseph A, David J Beerling, and Peter J Franks. 2010. {``Stomata: Key Players in the Earth System, Past and Present.''} \emph{Current Opinion in Plant Biology} 13 (3): 232--39. \url{https://doi.org/10.1016/j.pbi.2010.04.013}.

\leavevmode\vadjust pre{\hypertarget{ref-borsuk_structural_2022}{}}%
Borsuk, Aleca M., Adam B. Roddy, Guillaume Théroux‐Rancourt, and Craig R. Brodersen. 2022. {``Structural Organization of the Spongy Mesophyll.''} \emph{New Phytologist} 234 (3): 946--60. \url{https://doi.org/10.1111/nph.17971}.

\leavevmode\vadjust pre{\hypertarget{ref-brodribb_unified_2013}{}}%
Brodribb, Tim J., Greg J. Jordan, and Raymond J. Carpenter. 2013. {``Unified Changes in Cell Size Permit Coordinated Leaf Evolution.''} \emph{New Phytologist} 199 (2): 559--70. \url{https://doi.org/10.1111/nph.12300}.

\leavevmode\vadjust pre{\hypertarget{ref-bucher_stomatal_2017}{}}%
Bucher, Solveig Franziska, Karl Auerswald, Christina Grün-Wenzel, Steven I. Higgins, Javier Garcia Jorge, and Christine Römermann. 2017. {``Stomatal Traits Relate to Habitat Preferences of Herbaceous Species in a Temperate Climate.''} \emph{Flora} 229 (April): 107--15. \url{https://doi.org/10.1016/j.flora.2017.02.011}.

\leavevmode\vadjust pre{\hypertarget{ref-buckley_how_2015}{}}%
Buckley, Thomas N, Grace P John, Christine Scoffoni, and Lawren Sack. 2015. {``How Does Leaf Anatomy Influence Water Transport Outside the Xylem?''} \emph{Plant Physiology} 168: 1616--35.

\leavevmode\vadjust pre{\hypertarget{ref-burkner_brms_2017}{}}%
Bürkner, Paul-Christian. 2017. {``\textbf{Brms} : {An} \emph{r} {Package} for {Bayesian} {Multilevel} {Models} {Using} \emph{Stan}.''} \emph{Journal of Statistical Software} 80 (1). \url{https://doi.org/10.18637/jss.v080.i01}.

\leavevmode\vadjust pre{\hypertarget{ref-burkner_advanced_2018}{}}%
---------. 2018. {``Advanced {Bayesian} {Multilevel} {Modeling} with the {R} {Package} Brms.''} \emph{The R Journal} 10 (1): 395. \url{https://doi.org/10.32614/RJ-2018-017}.

\leavevmode\vadjust pre{\hypertarget{ref-casson_influence_2008}{}}%
Casson, Stuart, and Julie E. Gray. 2008. {``Influence of Environmental Factors on Stomatal Development.''} \emph{New Phytologist} 178 (1): 9--23. \url{https://doi.org/10.1111/j.1469-8137.2007.02351.x}.

\leavevmode\vadjust pre{\hypertarget{ref-chitwood_quantitative_2013}{}}%
Chitwood, Daniel H., Ravi Kumar, Lauren R. Headland, Aashish Ranjan, Michael F Covington, Yasunori Ichihashi, Daniel Fulop, et al. 2013. {``A Quantitative Genetic Basis for Leaf Morphology in a Set of Precisely Defined Tomato Introgression Lines.''} \emph{The Plant Cell} 25 (7): 2465--81.

\leavevmode\vadjust pre{\hypertarget{ref-chitwood_conflict_2012}{}}%
Chitwood, Daniel H., Daniel T. Naylor, Paradee Thammapichai, Axelle C. S. Weeger, Lauren R. Headland, and Neelima R. Sinha. 2012. {``Conflict Between {Intrinsic} {Leaf} {Asymmetry} and {Phyllotaxis} in the {Resupinate} {Leaves} of {Alstroemeria} Psittacina.''} \emph{Frontiers in Plant Science} 3. \url{https://doi.org/10.3389/fpls.2012.00182}.

\leavevmode\vadjust pre{\hypertarget{ref-clark_origin_2022}{}}%
Clark, James W., Brogan J. Harris, Alexander J. Hetherington, Natalia Hurtado-Castano, Robert A. Brench, Stuart Casson, Tom A. Williams, Julie E. Gray, and Alistair M. Hetherington. 2022. {``The Origin and Evolution of Stomata.''} \emph{Current Biology} 32 (11): R539--53. \url{https://doi.org/10.1016/j.cub.2022.04.040}.

\leavevmode\vadjust pre{\hypertarget{ref-conner_raissa_2014}{}}%
Conner, Jeffrey K., and Russell Lande. 2014. {``Raissa {L}. {Berg}'s Contributions to the Study of Phenotypic Integration, with a Professional Biographical Sketch.''} \emph{Philosophical Transactions of the Royal Society B: Biological Sciences} 369 (1649): 20130250. \url{https://doi.org/10.1098/rstb.2013.0250}.

\leavevmode\vadjust pre{\hypertarget{ref-de_boer_optimal_2016}{}}%
de Boer, Hugo J., Charles A. Price, Friederike Wagner‐Cremer, Stefan C. Dekker, Peter J. Franks, and Erik J. Veneklaas. 2016. {``Optimal Allocation of Leaf Epidermal Area for Gas Exchange.''} \emph{New Phytologist} 210 (4): 1219--28. \url{https://doi.org/10.1111/nph.13929}.

\leavevmode\vadjust pre{\hypertarget{ref-deans_optimization_2020}{}}%
Deans, Ross M., Timothy J. Brodribb, Florian A. Busch, and Graham D. Farquhar. 2020. {``Optimization Can Provide the Fundamental Link Between Leaf Photosynthesis, Gas Exchange and Water Relations.''} \emph{Nature Plants} 6 (9): 1116--25. \url{https://doi.org/10.1038/s41477-020-00760-6}.

\leavevmode\vadjust pre{\hypertarget{ref-dow_patterning_2014}{}}%
Dow, Graham J., and Dominique C. Bergmann. 2014. {``Patterning and Processes: How Stomatal Development Defines Physiological Potential.''} \emph{Current Opinion in Plant Biology} 21 (October): 67--74. \url{https://doi.org/10.1016/j.pbi.2014.06.007}.

\leavevmode\vadjust pre{\hypertarget{ref-dow_physiological_2014}{}}%
Dow, Graham J., Joseph A. Berry, and Dominique C. Bergmann. 2014. {``The Physiological Importance of Developmental Mechanisms That Enforce Proper Stomatal Spacing in \emph{{Arabidopsis} Thaliana}.''} \emph{New Phytologist} 201 (4): 1205--17. \url{https://doi.org/10.1111/nph.12586}.

\leavevmode\vadjust pre{\hypertarget{ref-dow_disruption_2017}{}}%
---------. 2017. {``Disruption of Stomatal Lineage Signaling or Transcriptional Regulators Has Differential Effects on Mesophyll Development, but Maintains Coordination of Gas Exchange.''} \emph{New Phytologist} 216 (1): 69--75. \url{https://doi.org/10.1111/nph.14746}.

\leavevmode\vadjust pre{\hypertarget{ref-drake_two_2019}{}}%
Drake, Paul L., Hugo J. de Boer, Stanislaus J. Schymanski, and Erik J. Veneklaas. 2019. {``Two Sides to Every Leaf: Water and {CO}\(_{\textrm{2}}\) Transport in Hypostomatous and Amphistomatous Leaves.''} \emph{New Phytologist} 222 (3): 1179--87. \url{https://doi.org/10.1111/nph.15652}.

\leavevmode\vadjust pre{\hypertarget{ref-drake_smaller_2013}{}}%
Drake, Paul L., Ray H Froend, and Peter J Franks. 2013. {``Smaller, Faster Stomata: Scaling of Stomatal Size, Rate of Response, and Stomatal Conductance.''} \emph{Journal of Experimental Botany} 64 (2): 495--505. \url{https://doi.org/10.1093/jxb/ers347}.

\leavevmode\vadjust pre{\hypertarget{ref-felsenstein_phylogenies_1985}{}}%
Felsenstein, Joseph. 1985. {``Phylogenies and the Comparative Method.''} \emph{The American Naturalist} 1 (125): 1--15.

\leavevmode\vadjust pre{\hypertarget{ref-ferris_leaf_2002}{}}%
Ferris, R., L. Long, S. M. Bunn, K. M. Robinson, H. D. Bradshaw, A. M. Rae, and G. Taylor. 2002. {``Leaf Stomatal and Epidermal Cell Development: Identification of Putative Quantitative Trait Loci in Relation to Elevated Carbon Dioxide Concentration in Poplar.''} \emph{Tree Physiology} 22 (9): 633--40. \url{https://doi.org/10.1093/treephys/22.9.633}.

\leavevmode\vadjust pre{\hypertarget{ref-fetter_growthdefense_2021}{}}%
Fetter, Karl C., David M. Nelson, and Stephen R. Keller. 2021. {``Growth‐defense Trade‐offs Masked in Unadmixed Populations Are Revealed by Hybridization.''} \emph{Evolution} 75 (6): 1450--65. \url{https://doi.org/10.1111/evo.14227}.

\leavevmode\vadjust pre{\hypertarget{ref-franks_maximum_2009}{}}%
Franks, Peter J, and David J Beerling. 2009. {``Maximum Leaf Conductance Driven by {CO}\(_{\textrm{2}}\) Effects on Stomatal Size and Density over Geologic Time.''} \emph{Proceedings of the National Academy of Sciences} 106 (25): 10343--47.

\leavevmode\vadjust pre{\hypertarget{ref-franks_effect_2001}{}}%
Franks, Peter J, and Graham D Farquhar. 2001. {``The {Effect} of {Exogenous} {Abscisic} {Acid} on {Stomatal} {Development}, {Stomatal} {Mechanics}, and {Leaf} {Gas} {Exchange} in \emph{{Tradescantia} Virginiana}.''} \emph{Plant Physiology} 125 (2): 935--42. \url{https://doi.org/10.1104/pp.125.2.935}.

\leavevmode\vadjust pre{\hypertarget{ref-gabry_cmdstanr_2022}{}}%
Gabry, Jonah, and Rok Češnovar. 2022. \emph{Cmdstanr: {R} {Interface} to '{CmdStan}'}. \href{https://mc-stan.org/cmdstanr,\%20https://discourse.mc-stan.org}{https://mc-stan.org/cmdstanr, https://discourse.mc-stan.org}.

\leavevmode\vadjust pre{\hypertarget{ref-galmes_leaf_2013}{}}%
Galmés, Jeroni, Joan Manuel Ochogavía, Jorge Gago, Emilio José Roldán, Josep Cifre, and Miquel Àngel Conesa. 2013. {``Leaf Responses to Drought Stress in {Mediterranean} Accessions of \emph{{Solanum} Lycopersicum} : Anatomical Adaptations in Relation to Gas Exchange Parameters: {Anatomical} Adaptations to Water Stress in Tomato.''} \emph{Plant, Cell \& Environment} 36 (5): 920--35. \url{https://doi.org/10.1111/pce.12022}.

\leavevmode\vadjust pre{\hypertarget{ref-gibson_structure-function_1996}{}}%
Gibson, Arthur C. 1996. \emph{Structure-{Function} {Relations} of {Warm} {Desert} {Plants}}. Berlin, Heidelberg: Springer Berlin / Heidelberg. \url{http://public.eblib.com/choice/PublicFullRecord.aspx?p=6495247}.

\leavevmode\vadjust pre{\hypertarget{ref-giuliani_coordination_2013}{}}%
Giuliani, Rita, Nuria Koteyeva, Elena Voznesenskaya, Marc A Evans, Asaph B Cousins, and Gerald E Edwards. 2013. {``Coordination of Leaf Photosynthesis, Transpiration, and Structural Traits in Rice and Wild Relatives (Genus \emph{o}ryza).''} \emph{Plant Physiology} 162 (3): 1632--51.

\leavevmode\vadjust pre{\hypertarget{ref-gutschick_photosynthesis_1984}{}}%
Gutschick, Vincent P. 1984. {``Photosynthesis Model for {C}\(_{\textrm{3}}\) Leaves Incorporating {CO}\(_{\textrm{2}}\) Transport, Propagation of Radiation, and Biochemistry 1. Kinetics and Their Parameterization.''} \emph{Photosynthetica} 18 (4): 549--68.

\leavevmode\vadjust pre{\hypertarget{ref-hall_relationships_2005}{}}%
Hall, N. M., H. Griffiths, J. A. Corlett, H. G. Jones, J. Lynn, and G. J. King. 2005. {``Relationships Between Water-Use Traits and Photosynthesis in \emph{{Brassica} Oleracea} Resolved by Quantitative Genetic Analysis.''} \emph{Plant Breeding} 124 (6): 557--64. \url{https://doi.org/10.1111/j.1439-0523.2005.01164.x}.

\leavevmode\vadjust pre{\hypertarget{ref-hansen_is_2003}{}}%
Hansen, Thomas F. 2003. {``Is Modularity Necessary for Evolvability?''} \emph{Biosystems} 69 (2-3): 83--94. \url{https://doi.org/10.1016/S0303-2647(02)00132-6}.

\leavevmode\vadjust pre{\hypertarget{ref-harrison_influence_2020}{}}%
Harrison, Emily L., Lucia Arce Cubas, Julie E. Gray, and Christopher Hepworth. 2020. {``The Influence of Stomatal Morphology and Distribution on Photosynthetic Gas Exchange.''} \emph{The Plant Journal} 101 (4): 768--79. \url{https://doi.org/10.1111/tpj.14560}.

\leavevmode\vadjust pre{\hypertarget{ref-haworth_co-ordination_2013}{}}%
Haworth, Matthew, Caroline Elliott-Kingston, and Jennifer C. McElwain. 2013. {``Co-Ordination of Physiological and Morphological Responses of Stomata to Elevated {[}{CO2}{]} in Vascular Plants.''} \emph{Oecologia} 171 (1): 71--82. \url{https://doi.org/10.1007/s00442-012-2406-9}.

\leavevmode\vadjust pre{\hypertarget{ref-henry_stomatal_2019}{}}%
Henry, Christian, Grace P. John, Ruihua Pan, Megan K. Bartlett, Leila R. Fletcher, Christine Scoffoni, and Lawren Sack. 2019. {``A Stomatal Safety-Efficiency Trade-Off Constrains Responses to Leaf Dehydration.''} \emph{Nature Communications} 10 (1): 3398. \url{https://doi.org/10.1038/s41467-019-11006-1}.

\leavevmode\vadjust pre{\hypertarget{ref-hetherington_role_2003}{}}%
Hetherington, Alistair M., and F. Ian Woodward. 2003. {``The Role of Stomata in Sensing and Driving Environmental Change.''} \emph{Nature} 424 (6951): 901--8. \url{https://doi.org/10.1038/nature01843}.

\leavevmode\vadjust pre{\hypertarget{ref-ishimaru_identification_2001}{}}%
Ishimaru, Ken, Kanako Shirota, Masae Higa, and Yoshinobu Kawamitsu. 2001. {``Identification of Quantitative Trait Loci for Adaxial and Abaxial Stomatal Frequencies in \emph{{Oryza} Sativa}.''} \emph{Plant Physiology and Biochemistry} 39 (2): 173--77. \url{https://doi.org/10.1016/S0981-9428(00)01232-8}.

\leavevmode\vadjust pre{\hypertarget{ref-jordan_using_2014}{}}%
Jordan, Gregory J., Raymond J. Carpenter, and Timothy J. Brodribb. 2014. {``Using Fossil Leaves as Evidence for Open Vegetation.''} \emph{Palaeogeography, Palaeoclimatology, Palaeoecology} 395 (February): 168--75. \url{https://doi.org/10.1016/j.palaeo.2013.12.035}.

\leavevmode\vadjust pre{\hypertarget{ref-jordan_environmental_2015}{}}%
Jordan, Gregory J., Raymond J. Carpenter, Anthony Koutoulis, Aina Price, and Timothy J. Brodribb. 2015. {``Environmental Adaptation in Stomatal Size Independent of the Effects of Genome Size.''} \emph{New Phytologist} 205 (2): 608--17. \url{https://doi.org/10.1111/nph.13076}.

\leavevmode\vadjust pre{\hypertarget{ref-kaluthota_higher_2015}{}}%
Kaluthota, Sobadini, David W. Pearce, Luke M. Evans, Matthew G. Letts, Thomas G. Whitham, and Stewart B. Rood. 2015. {``Higher Photosynthetic Capacity from Higher Latitude: Foliar Characteristics and Gas Exchange of Southern, Central and Northern Populations of \emph{{Populus} Angustifolia}.''} Edited by David Tissue. \emph{Tree Physiology} 35 (9): 936--48. \url{https://doi.org/10.1093/treephys/tpv069}.

\leavevmode\vadjust pre{\hypertarget{ref-kelly_plant_1995}{}}%
Kelly, C. K., and D. J. Beerling. 1995. {``Plant {Life} {Form}, {Stomatal} {Density} and {Taxonomic} {Relatedness}: {A} {Reanalysis} of {Salisbury} (1927).''} \emph{Functional Ecology} 9 (3): 422. \url{https://doi.org/10.2307/2390005}.

\leavevmode\vadjust pre{\hypertarget{ref-kidner_signaling_2010}{}}%
Kidner, Catherine A., and Marja C. P. Timmermans. 2010. {``Signaling {Sides}.''} In \emph{Current {Topics} in {Developmental} {Biology}}, 91:141--68. Elsevier. \url{https://doi.org/10.1016/S0070-2153(10)91005-3}.

\leavevmode\vadjust pre{\hypertarget{ref-lake_longdistance_2002}{}}%
Lake, Janice A., F. Ian Woodward, and W. Paul Quick. 2002. {``Long‐distance {CO2} Signalling in Plants.''} \emph{Journal of Experimental Botany} 53 (367): 183--93. \url{https://doi.org/10.1093/jexbot/53.367.183}.

\leavevmode\vadjust pre{\hypertarget{ref-lande_quantitative_1979}{}}%
Lande, Russell. 1979. {``Quantitative {Genetic} {Analysis} of {Multivariate} {Evolution}, {Applied} to {Brain}: {Body} {Size} {Allometry}.''} \emph{Evolution} 33 (1): 402. \url{https://doi.org/10.2307/2407630}.

\leavevmode\vadjust pre{\hypertarget{ref-lange_robust_1989}{}}%
Lange, Kenneth L., Roderick J. A. Little, and Jeremy M. G. Taylor. 1989. {``Robust {Statistical} {Modeling} {Using} the t {Distribution}.''} \emph{Journal of the American Statistical Association} 84 (408): 881--96. \url{https://doi.org/10.1080/01621459.1989.10478852}.

\leavevmode\vadjust pre{\hypertarget{ref-lawson_guard_2020}{}}%
Lawson, Tracy, and Jack Matthews. 2020. {``Guard {Cell} {Metabolism} and {Stomatal} {Function}.''} \emph{Annual Review of Plant Biology} 71 (1): 273--302. \url{https://doi.org/10.1146/annurev-arplant-050718-100251}.

\leavevmode\vadjust pre{\hypertarget{ref-laza_quantitative_2010}{}}%
Laza, Ma. Rebecca C., Motohiko Kondo, Osamu Ideta, Edward Barlaan, and Tokio Imbe. 2010. {``Quantitative Trait Loci for Stomatal Density and Size in Lowland Rice.''} \emph{Euphytica} 172 (2): 149--58. \url{https://doi.org/10.1007/s10681-009-0011-8}.

\leavevmode\vadjust pre{\hypertarget{ref-lehmann_effects_2015}{}}%
Lehmann, Peter, and Dani Or. 2015. {``Effects of Stomata Clustering on Leaf Gas Exchange.''} \emph{New Phytologist} 207 (4): 1015--25. \url{https://doi.org/10.1111/nph.13442}.

\leavevmode\vadjust pre{\hypertarget{ref-leitch_angiosperm_2019}{}}%
Leitch, Ilia J., E Johnston, Jaume Pellicer, Oriane Hidalgo, and Michael D. Bennett. 2019. {``Angiosperm {DNA} {C}-Values Database.''} \emph{Angiosperm DNA C-Values Database}. \url{https://cvalues.science.kew.org/}.

\leavevmode\vadjust pre{\hypertarget{ref-liu_variation_2018}{}}%
Liu, Congcong, Nianpeng He, Jiahui Zhang, Ying Li, Qiufeng Wang, Lawren Sack, and Guirui Yu. 2018. {``Variation of Stomatal Traits from Cold Temperate to Tropical Forests and Association with Water Use Efficiency.''} Edited by Shuli Niu. \emph{Functional Ecology} 32 (1): 20--28. \url{https://doi.org/10.1111/1365-2435.12973}.

\leavevmode\vadjust pre{\hypertarget{ref-liu_scaling_2021}{}}%
Liu, Congcong, Christopher D Muir, Ying Li, Li Xu, Mingxu Li, Jiahui Zhang, Hugo Jan de Boer, et al. 2021. {``Scaling Between Stomatal Size and Density in Forest Plants.''} Preprint. Plant Biology. \url{https://doi.org/10.1101/2021.04.25.441252}.

\leavevmode\vadjust pre{\hypertarget{ref-lundgren_mesophyll_2019}{}}%
Lundgren, Marjorie R., Andrew Mathers, Alice L. Baillie, Jessica Dunn, Matthew J. Wilson, Lee Hunt, Radoslaw Pajor, et al. 2019. {``Mesophyll Porosity Is Modulated by the Presence of Functional Stomata.''} \emph{Nature Communications} 10 (1): 2825. \url{https://doi.org/10.1038/s41467-019-10826-5}.

\leavevmode\vadjust pre{\hypertarget{ref-lynch_genetics_1998}{}}%
Lynch, Michael, and Bruce Walsh. 1998. \emph{Genetics and Analysis of Quantitative Traits}. Sunderland, Mass: Sinauer.

\leavevmode\vadjust pre{\hypertarget{ref-mackenzie_proving_1999}{}}%
Mackenzie, Dana. 1999. {``Proving the {Perfection} of the {Honeycomb}.''} \emph{Science} 285 (5432): 1338--39. \url{https://doi.org/10.1126/science.285.5432.1338}.

\leavevmode\vadjust pre{\hypertarget{ref-magallon_metacalibrated_2015}{}}%
Magallón, Susana, Sandra Gómez-Acevedo, Luna L. Sánchez-Reyes, and Tania Hernández-Hernández. 2015. {``A Metacalibrated Time-Tree Documents the Early Rise of Flowering Plant Phylogenetic Diversity.''} \emph{New Phytologist} 207 (2): 437--53. \url{https://doi.org/10.1111/nph.13264}.

\leavevmode\vadjust pre{\hypertarget{ref-maritan_optimal_2000}{}}%
Maritan, Amos, Cristian Micheletti, Antonio Trovato, and Jayanth R. Banavar. 2000. {``Optimal Shapes of Compact Strings.''} \emph{Nature} 406 (6793): 287--90. \url{https://doi.org/10.1038/35018538}.

\leavevmode\vadjust pre{\hypertarget{ref-maynard_smith_developmental_1985}{}}%
Maynard Smith, John, R. Burian, S. Kauffman, P. Alberch, J. Campbell, B. Goodwin, R. Lande, D. Raup, and L. Wolpert. 1985. {``Developmental {Constraints} and {Evolution}: {A} {Perspective} from the {Mountain} {Lake} {Conference} on {Development} and {Evolution}.''} \emph{The Quarterly Review of Biology} 60 (3): 265--87. \url{http://www.jstor.org/stable/2828504}.

\leavevmode\vadjust pre{\hypertarget{ref-mcghee_theoretical_1999}{}}%
McGhee, George R. 1999. \emph{Theoretical Morphology: The Concept and Its Applications}. Perspectives in Paleobiology and Earth History. New York: Columbia University Press.

\leavevmode\vadjust pre{\hypertarget{ref-mcghee_geometry_2007}{}}%
---------. 2007. \emph{The Geometry of Evolution: Adaptive Landscapes and Theoretical Morphospaces}. Cambridge, UK ; New York: Cambridge University Press.

\leavevmode\vadjust pre{\hypertarget{ref-mckown_association_2014}{}}%
McKown, Athena D., Robert D. Guy, Linda Quamme, Jaroslav Klápště, Jonathan La Mantia, C. P. Constabel, Yousry A. El-Kassaby, Richard C. Hamelin, Michael Zifkin, and M. S. Azam. 2014. {``Association Genetics, Geography and Ecophysiology Link Stomatal Patterning in \emph{{Populus} Trichocarpa} with Carbon Gain and Disease Resistance Trade-Offs.''} \emph{Molecular Ecology} 23 (23): 5771--90. \url{https://doi.org/10.1111/mec.12969}.

\leavevmode\vadjust pre{\hypertarget{ref-mckown_role_2019}{}}%
McKown, Athena D., Jaroslav Klápště, Robert D. Guy, Oliver R. A. Corea, Steffi Fritsche, Jürgen Ehlting, Yousry A. El‐Kassaby, and Shawn D. Mansfield. 2019. {``A Role for \emph{SPEECHLESS} in the Integration of Leaf Stomatal Patterning with the Growth Vs Disease Trade‐off in Poplar.''} \emph{New Phytologist} 223 (4): 1888--1903. \url{https://doi.org/10.1111/nph.15911}.

\leavevmode\vadjust pre{\hypertarget{ref-metcalfe_anatomy_1950}{}}%
Metcalfe, Charles Russell, and Laurence Chalk. 1950. \emph{Anatomy of the Dicotyledons, {Vols}. 1 \& 2}. First. Oxford: Oxford University Press.

\leavevmode\vadjust pre{\hypertarget{ref-mitchison_phyllotaxis_1977}{}}%
Mitchison, G. J. 1977. {``Phyllotaxis and the {Fibonacci} {Series}: {An} Explanation Is Offered for the Characteristic Spiral Leaf Arrangement Found in Many Plants.''} \emph{Science} 196 (4287): 270--75. \url{https://doi.org/10.1126/science.196.4287.270}.

\leavevmode\vadjust pre{\hypertarget{ref-mott_adaptive_1982}{}}%
Mott, Keith A., Arthur C. Gibson, and James W. O'Leary. 1982. {``The Adaptive Significance of Amphistomatic Leaves.''} \emph{Plant, Cell \& Environment} 5 (6): 455--60. \url{https://doi.org/10.1111/1365-3040.ep11611750}.

\leavevmode\vadjust pre{\hypertarget{ref-muir_making_2015}{}}%
Muir, Christopher D. 2015. {``Making Pore Choices: Repeated Regime Shifts in Stomatal Ratio.''} \emph{Proceedings of the Royal Society B: Biological Sciences} 282 (1813): 20151498. \url{https://doi.org/10.1098/rspb.2015.1498}.

\leavevmode\vadjust pre{\hypertarget{ref-muir_light_2018}{}}%
---------. 2018. {``Light and Growth Form Interact to Shape Stomatal Ratio Among {British} Angiosperms.''} \emph{New Phytologist} 218 (1): 242--52.

\leavevmode\vadjust pre{\hypertarget{ref-muir_is_2019}{}}%
---------. 2019. {``Is {Amphistomy} an {Adaptation} to {High} {Light}? {Optimality} {Models} of {Stomatal} {Traits} Along {Light} {Gradients}.''} \emph{Integrative and Comparative Biology} 59 (3): 571--84. \url{https://doi.org/10.1093/icb/icz085}.

\leavevmode\vadjust pre{\hypertarget{ref-muir_data_2022}{}}%
Muir, Christopher D, Miquel Àngel Conesa, Jeroni Galmés, Varsha S Pathare, Patricia Rivera, Rosana López Rodríguez, Teresa Terrazas, and Dongliang Xiong. 2022. {``Data from: {How} Important Are Functional and Developmental Constraints on Phenotypic Evolution? {An} Empirical Test with the Stomatal Anatomy of Flowering Plants.''} \url{https://doi.org/10.5061/dryad.0rxwdbs42}.

\leavevmode\vadjust pre{\hypertarget{ref-muir_unpublished_2022}{}}%
Muir, Christopher D, Jeroni Galmés, and Miquel À Conesa. 2022. {``Unpublished Data.''}

\leavevmode\vadjust pre{\hypertarget{ref-muir_quantitative_2014}{}}%
Muir, Christopher D, James B Pease, and Leonie C Moyle. 2014. {``Quantitative {Genetic} {Analysis} {Indicates} {Natural} {Selection} on {Leaf} {Phenotypes} {Across} {Wild} {Tomato} {Species} (\emph{Solanum} Sect. \emph{Lycopersicon} ; {Solanaceae}).''} \emph{Genetics} 198 (4): 1629--43. \url{https://doi.org/10.1534/genetics.114.169276}.

\leavevmode\vadjust pre{\hypertarget{ref-murray_consistent_2020}{}}%
Murray, Michelle, Wuu Kuang Soh, Charilaos Yiotis, Robert A. Spicer, Tracy Lawson, and Jennifer C. McElwain. 2020. {``Consistent {Relationship} Between {Field}-{Measured} {Stomatal} {Conductance} and {Theoretical} {Maximum} {Stomatal} {Conductance} in {C} \(_{\textrm{3}}\) {Woody} {Angiosperms} in {Four} {Major} {Biomes}.''} \emph{International Journal of Plant Sciences} 181 (1): 142--54. \url{https://doi.org/10.1086/706260}.

\leavevmode\vadjust pre{\hypertarget{ref-niklas_role_1988}{}}%
Niklas, Karl J. 1988. {``The {Role} of {Phyllotatic} {Pattern} as a "{Developmental} {Constraint}" {On} the {Interception} of {Light} by {Leaf} {Surfaces}.''} \emph{Evolution} 42 (1): 1. \url{https://doi.org/10.2307/2409111}.

\leavevmode\vadjust pre{\hypertarget{ref-olson_plant_2019}{}}%
Olson, Mark E. 2019. {``Plant {Evolutionary} {Ecology} in the {Age} of the {Extended} {Evolutionary} {Synthesis}.''} \emph{Integrative and Comparative Biology} 59 (3): 493--502. \url{https://doi.org/10.1093/icb/icz042}.

\leavevmode\vadjust pre{\hypertarget{ref-olson_how_2015}{}}%
Olson, Mark E., and Alfonso Arroyo-Santos. 2015. {``How to {Study} {Adaptation} (and {Why} {To} {Do} {It} {That} {Way}).''} \emph{The Quarterly Review of Biology} 90 (2): 167--91. \url{https://doi.org/10.1086/681438}.

\leavevmode\vadjust pre{\hypertarget{ref-parkhurst_adaptive_1978}{}}%
Parkhurst, David F. 1978. {``The {Adaptive} {Significance} of {Stomatal} {Occurrence} on {One} or {Both} {Surfaces} of {Leaves}.''} \emph{The Journal of Ecology} 66 (2): 367. \url{https://doi.org/10.2307/2259142}.

\leavevmode\vadjust pre{\hypertarget{ref-parkhurst_intercellular_1990}{}}%
Parkhurst, David F., and Keith A. Mott. 1990. {``Intercellular Diffusion Limits to {CO}\(_{\textrm{2}}\) Uptake in Leaves: Studies in Air and Helox.''} \emph{Plant Physiology} 94 (3): 1024--32. \url{https://doi.org/10.1104/pp.94.3.1024}.

\leavevmode\vadjust pre{\hypertarget{ref-pathare_increased_2020}{}}%
Pathare, Varsha S., Nuria Koteyeva, and Asaph B. Cousins. 2020. {``Increased Adaxial Stomatal Density Is Associated with Greater Mesophyll Surface Area Exposed to Intercellular Air Spaces and Mesophyll Conductance in Diverse {C} \(_{\textrm{4}}\) Grasses.''} \emph{New Phytologist} 225 (1): 169--82. \url{https://doi.org/10.1111/nph.16106}.

\leavevmode\vadjust pre{\hypertarget{ref-peat_comparative_1994}{}}%
Peat, H J, and Alistair H Fitter. 1994. {``A Comparative Study of the Distribution and Density of Stomata in the {British} Flora.''} \emph{Biological Journal of the Linnean Society} 52 (4): 377--93.

\leavevmode\vadjust pre{\hypertarget{ref-pelabon_evolution_2014}{}}%
Pélabon, Christophe, Cyril Firmat, Geir H. Bolstad, Kjetil L. Voje, David Houle, Jason Cassara, Arnaud Le Rouzic, and Thomas F. Hansen. 2014. {``Evolution of Morphological Allometry: {The} Evolvability of Allometry.''} \emph{Annals of the New York Academy of Sciences} 1320 (1): 58--75. \url{https://doi.org/10.1111/nyas.12470}.

\leavevmode\vadjust pre{\hypertarget{ref-pellicer_plant_2020}{}}%
Pellicer, Jaume, and Ilia J. Leitch. 2020. {``The {Plant} {DNA} {C}‐values Database (Release 7.1): An Updated Online Repository of Plant Genome Size Data for Comparative Studies.''} \emph{New Phytologist} 226 (2): 301--5. \url{https://doi.org/10.1111/nph.16261}.

\leavevmode\vadjust pre{\hypertarget{ref-pillitteri_mechanisms_2012}{}}%
Pillitteri, Lynn Jo, and Keiko U. Torii. 2012. {``Mechanisms of {Stomatal} {Development}.''} \emph{Annual Review of Plant Biology} 63 (1): 591--614. \url{https://doi.org/10.1146/annurev-arplant-042811-105451}.

\leavevmode\vadjust pre{\hypertarget{ref-porth_evolutionary_2015}{}}%
Porth, Ilga, Jaroslav Klápště, Athena D McKown, Jonathan La Mantia, Robert D Guy, Pär K Ingvarsson, Richard Hamelin, et al. 2015. {``Evolutionary {Quantitative} {Genomics} of {Populus} Trichocarpa.''} \emph{PLOS ONE}, 25.

\leavevmode\vadjust pre{\hypertarget{ref-r_core_team_r:_2022}{}}%
R Core Team. 2022. \emph{R: {A} {Language} and {Environment} for {Statistical} {Computing}}. Vienna, Austria: R Foundation for Statistical Computing. \url{http://www.R-project.org/}.

\leavevmode\vadjust pre{\hypertarget{ref-rae_elucidating_2006}{}}%
Rae, A. M., Rachel Ferris, M. J. Tallis, and Gail Taylor. 2006. {``Elucidating Genomic Regions Determining Enhanced Leaf Growth and Delayed Senescence in Elevated {CO2}.''} \emph{Plant, Cell and Environment} 29 (9): 1730--41. \url{https://doi.org/10.1111/j.1365-3040.2006.01545.x}.

\leavevmode\vadjust pre{\hypertarget{ref-raup_geometric_1966}{}}%
Raup, David M. 1966. {``Geometric {Analysis} of {Shell} {Coiling}: {General} {Problems}.''} \emph{Journal of Paleontology} 40 (5): 1178--90. \url{https://www.jstor.org/stable/1301992}.

\leavevmode\vadjust pre{\hypertarget{ref-reinhardt_law_2022}{}}%
Reinhardt, Didier, and Edyta M. Gola. 2022. {``Law and Order in Plants -- the Origin and Functional Relevance of Phyllotaxis.''} \emph{Trends in Plant Science}, May, S1360138522001261. \url{https://doi.org/10.1016/j.tplants.2022.04.005}.

\leavevmode\vadjust pre{\hypertarget{ref-roddy_uncorrelated_2013}{}}%
Roddy, Adam B., C. Matt Guilliams, Terapan Lilittham, Jessica Farmer, Vanessa Wormser, Trang Pham, Paul V. A. Fine, Taylor S. Feild, and Todd E. Dawson. 2013. {``Uncorrelated Evolution of Leaf and Petal Venation Patterns Across the Angiosperm Phylogeny.''} \emph{Journal of Experimental Botany} 64 (13): 4081--88. \url{https://doi.org/10.1093/jxb/ert247}.

\leavevmode\vadjust pre{\hypertarget{ref-roddy_scaling_2020}{}}%
Roddy, Adam B., Guillaume Théroux-Rancourt, Tito Abbo, Joseph W. Benedetti, Craig R. Brodersen, Mariana Castro, Silvia Castro, et al. 2020. {``The {Scaling} of {Genome} {Size} and {Cell} {Size} {Limits} {Maximum} {Rates} of {Photosynthesis} with {Implications} for {Ecological} {Strategies}.''} \emph{International Journal of Plant Sciences} 181 (1): 75--87. \url{https://doi.org/10.1086/706186}.

\leavevmode\vadjust pre{\hypertarget{ref-royer_stomatal_2001}{}}%
Royer, D. L. 2001. {``Stomatal Density and Stomatal Index as Indicators of Paleoatmospheric {CO}\(_{\textrm{2}}\) Concentration.''} \emph{Review of Palaeobotany and Palynology} 114 (1-2): 1--28. \url{https://doi.org/10.1016/S0034-6667(00)00074-9}.

\leavevmode\vadjust pre{\hypertarget{ref-sack_developmental_2016}{}}%
Sack, Lawren, and Thomas N Buckley. 2016. {``The Developmental Basis of Stomatal Density and Flux.''} \emph{Plant Physiology} 171 (4): 2358--63. \url{https://doi.org/10.1104/pp.16.00476}.

\leavevmode\vadjust pre{\hypertarget{ref-sack_hydrology_2003}{}}%
Sack, Lawren, P. D. Cowan, N. Jaikumar, and N. M. Holbrook. 2003. {``The {`Hydrology'} of Leaves: Co-Ordination of Structure and Function in Temperate Woody Species.''} \emph{Plant, Cell \& Environment} 26 (8): 1343--56. \url{https://doi.org/10.1046/j.0016-8025.2003.01058.x}.

\leavevmode\vadjust pre{\hypertarget{ref-salisbury_i_1928}{}}%
Salisbury, Edward James. 1928. {``I. {On} the Causes and Ecological Significance of Stomatal Frequency, with Special Reference to the Woodland Flora.''} \emph{Philosophical Transactions of the Royal Society of London. Series B, Containing Papers of a Biological Character} 216 (431-439): 1--65. \url{https://doi.org/10.1098/rstb.1928.0001}.

\leavevmode\vadjust pre{\hypertarget{ref-schluter_adaptive_1996}{}}%
Schluter, Dolph. 1996. {``Adaptive {Radiation} {Along} {Genetic} {Lines} of {Least} {Resistance}.''} \emph{Evolution} 50 (5): 1766. \url{https://doi.org/10.2307/2410734}.

\leavevmode\vadjust pre{\hypertarget{ref-simonin_genome_2018}{}}%
Simonin, Kevin A., and Adam B. Roddy. 2018. {``Genome Downsizing, Physiological Novelty, and the Global Dominance of Flowering Plants.''} Edited by Andrew Tanentzap. \emph{PLOS Biology} 16 (1): e2003706. \url{https://doi.org/10.1371/journal.pbio.2003706}.

\leavevmode\vadjust pre{\hypertarget{ref-simova_geometrical_2012}{}}%
Šímová, Irena, and Tomáš Herben. 2012. {``Geometrical Constraints in the Scaling Relationships Between Genome Size, Cell Size and Cell Cycle Length in Herbaceous Plants.''} \emph{Proceedings of the Royal Society B: Biological Sciences} 279 (1730): 867--75. \url{https://doi.org/10.1098/rspb.2011.1284}.

\leavevmode\vadjust pre{\hypertarget{ref-smith_constructing_2018}{}}%
Smith, Stephen A., and Joseph W. Brown. 2018. {``Constructing a Broadly Inclusive Seed Plant Phylogeny.''} \emph{American Journal of Botany} 105 (3): 302--14. \url{https://doi.org/10.1002/ajb2.1019}.

\leavevmode\vadjust pre{\hypertarget{ref-smith_associations_1998}{}}%
Smith, William K., David T. Bell, and Kelly A. Shepherd. 1998. {``Associations Between Leaf Structure, Orientation, and Sunlight Exposure in Five {Western} {Australian} Communities.''} \emph{American Journal of Botany} 85 (1): 51--63.

\leavevmode\vadjust pre{\hypertarget{ref-stan_development_team_stan_2022}{}}%
Stan Development Team. 2022. \emph{Stan {Modeling} {Language} {Users} {Guide} and {Reference} {Manual}}. \url{https://mc-stan.org}.

\leavevmode\vadjust pre{\hypertarget{ref-theroux-rancourt_maximum_2021}{}}%
Théroux-Rancourt, Guillaume, Adam B. Roddy, J. Mason Earles, Matthew E. Gilbert, Maciej A. Zwieniecki, C. Kevin Boyce, Danny Tholen, Andrew J. McElrone, Kevin A. Simonin, and Craig R. Brodersen. 2021. {``Maximum {CO} \(_{\textrm{2}}\) Diffusion Inside Leaves Is Limited by the Scaling of Cell Size and Genome Size.''} \emph{Proceedings of the Royal Society B: Biological Sciences} 288 (1945): 20203145. \url{https://doi.org/10.1098/rspb.2020.3145}.

\leavevmode\vadjust pre{\hypertarget{ref-vehtari_rank-normalization_2021}{}}%
Vehtari, Aki, Andrew Gelman, Daniel Simpson, Bob Carpenter, and Paul-Christian Bürkner. 2021. {``Rank-{Normalization}, {Folding}, and {Localization}: {An} {Improved} \emph{r} for {Assessing} {Convergence} of {MCMC} (with {Discussion}).''} \emph{Bayesian Analysis} 16 (2). \url{https://doi.org/10.1214/20-BA1221}.

\leavevmode\vadjust pre{\hypertarget{ref-weiss_untersuchungen_1865}{}}%
Weiss, Adolph. 1865. {``Untersuchungen Über Die {Zahlen}- Und {Grössenverhältnisse} Der {Spaltöffnungen}.''} \emph{Jahrbücher Für Wissenschaftliche Botanik} 4: 125--96.

\leavevmode\vadjust pre{\hypertarget{ref-woodward_stomatal_1987}{}}%
Woodward, F Ian. 1987. {``Stomatal Numbers Are Sensitive to Increases in {CO}\(_{\textrm{2}}\) from Pre-Industrial Levels.''} \emph{Nature} 327 (6123): 617--18.

\leavevmode\vadjust pre{\hypertarget{ref-zanne_three_2014}{}}%
Zanne, Amy E., David C. Tank, William K. Cornwell, Jonathan M. Eastman, Stephen A. Smith, Richard G. FitzJohn, Daniel J. McGlinn, et al. 2014. {``Three Keys to the Radiation of Angiosperms into Freezing Environments.''} \emph{Nature} 506 (7486): 89--92. \url{https://doi.org/10.1038/nature12872}.

\leavevmode\vadjust pre{\hypertarget{ref-zeiger_stomatal_1987}{}}%
Zeiger, Eduardo, G. D. Farquhar, and I. R. Cowan, eds. 1987. \emph{Stomatal Function}. Stanford, Calif: Stanford University Press.

\end{CSLReferences}

\end{document}
